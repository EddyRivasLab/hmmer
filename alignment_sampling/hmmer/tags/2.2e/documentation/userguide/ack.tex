\chapter {Acknowledgements and history}

\begin{quote}
\textit{
It must be remembered that there is nothing more difficult to plan, more
doubtful of success, nor more dangerous to manage, than the creation of
a new system.  For the initiator has the emnity of all who would profit
by the preservation of the old institutions and merely lukewarm defenders
in those who would gain by the new ones.\\}
\hspace*{\fill} -- Niccolo Machiavelli
\end{quote}

HMMER 1 was developed at the MRC Laboratory of Molecular Biology,
Cambridge UK, while I was a postdoc with Dr. Richard Durbin. I thank
the Human Frontier Science Program and the National Institutes of
Health for their support. 

HMMER 1.8 (and subsequent minor releases) was the first public release
of HMMER in April 1995. A number of modifications and improvements
went into HMMER 1.9 code, but 1.9 was never released. Some versions of
HMMER 1.9 did inadvertently escape St. Louis and make it to other
sites, but it was never documented or supported. HMMER 1.9 collapsed
under its own weight in 1996, when the number of ugly hacks increased
to a critical mass.

HMMER 2 is a nearly complete rewrite, based on a new model
architecture dubbed ``Plan 7''. Implementation was begun in November
1996 at Washington University in St. Louis. I thank the Washington
University Dept. of Genetics, the NIH National Human Genome Research
Institute, and Monsanto for their support during this (extremely
stressful) time. There is nothing like throwing four years of work
away and starting fresh to make you question your sanity. Working on a
book \cite{Durbin98} and starting up a lab at the same time made it
all doubly ``exciting''. If you are so bored as to actually read the
code, you will run across inexplicable comments that note where I was
when I wrote various parts, making up a sort of disjointed diary of
this period; amongst other places, parts of HMMER 2 were written in
airport lounges, on TWA flights 720 and 721 to and from London, in
Graeme Mitchison's kitchen in Cambridge, and on vacations while my
(ex-)wife wasn't watching. I therefore owe special thanks (I think) to
the Biochemistry Academic Contacts Committee at Eli Lilly \& Co. for a
gift that paid for the trusty Linux laptop on which much of HMMER 2
was written.

The MRC-LMB computational molecular biology discussion group has
contributed many ideas to HMMER. In particular, I thank Richard
Durbin, Graeme Mitchison, Erik Sonnhammer, Alex Bateman, Ewan Birney,
Gos Micklem, Tim Hubbard, Roger Sewall, David MacKay, and Cyrus
Chothia. Any errors in the code, though, are my fault alone, of
course.

Sequence format parsing ({\tt sqio.c}) in HMMER is derived from an
early release of the {\tt READSEQ} package by Don Gilbert, Indiana
University. Thanks to Don for an excellent piece of software; and
apologies for the mangling I've put it through since.  The file {\tt
hsregex.c} is a derivative of Henry Spencer's regular expression
library; thanks, Henry. Several miscellaneous functions in {\tt
sre\_math.c} are taken from public domain sources and are credited in
the code's comments. {\tt masks.c} includes a modified copy of the XNU
source code from David States and Jean-Michel Claverie.

In many other places, I've reimplemented algorithms described in the
literature. These are too numerous to credit and thank here. The
original references are given in the comments of the code. However,
I've borrowed (stolen?) more than once from the following folks that
I'd like to be sure to thank: Steve Altschul, Pierre Baldi, Phillip
Bucher, Warren Gish, David Haussler, Steve and Jorja Henikoff, Richard
Hughey, Kevin Karplus, Anders Krogh, Bill Pearson, and Kimmen
Sjolander.

HMMER is developed on Silicon Graphics and Linux machines in my lab.
Richard Durbin at the Sanger Centre (Hinxton UK) and LaDeana Hillier
and Warren Gish at the WashU Genome Sequencing Center have generously
donated time on Sun, Intel/Solaris, and DEC Alpha platforms for
additional development work. I thank Silicon Graphics, especially Juli
Nash and Wade McLain, for their continued support and assistance. Dave
Cortesi at SGI contributed much useful advice on the POSIX threads
implementation. I am proud to acknowledge a tremendous debt to the
development tools that I use from the free software community: an
incomplete list includes GNU gcc, gdb, emacs, and autoconf; Cygnus'
and others' egcs compiler; Conor Cahill's dbmalloc library; Bruce
Perens' ElectricFence; Walter Tichy's RCS; Larry Wall's perl; LaTeX
and TeX from Leslie Lamport and Don Knuth; Nikos Drakos' latex2html;
Thomas Phelps' PolyglotMan; Linus Torvalds' Linux operating system;
and the folks at Red Hat Linux.

Finally, I'd like to cryptically thank Dave ``Mr. Frog'' Pare and Tom
``Chainsaw'' Ruschak for a totally unrelated free software product
that was historically instrumental in HMMER's development, for reasons
that are best not discussed while sober.
