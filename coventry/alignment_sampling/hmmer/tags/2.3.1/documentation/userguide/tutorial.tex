\section{Tutorial}
\label{section:tutorial}

Here's a tutorial walk-through of some small projects with HMMER.
This section should be sufficient to get you started on work of your
own, and you can (at least temporarily) skip the rest of the Guide.

\subsection {The programs in HMMER}

There are currently nine programs supported in the HMMER 2 package:

\begin{wideitem}
\item[\emprog{hmmalign}] Align sequences to an existing model.
\item[\emprog{hmmbuild}] Build a model from a multiple sequence alignment.
\item[\emprog{hmmcalibrate}] Takes an HMM and empirically determines
parameters that are used to make searches more sensitive, by
calculating more accurate expectation value scores (E-values).
\item[\emprog{hmmconvert}] Convert a model file into different formats,
including a compact HMMER 2 binary format, and ``best effort''
emulation of GCG profiles.
\item[\emprog{hmmemit}] Emit sequences probabilistically from a profile HMM.
\item[\emprog{hmmfetch}] Get a single model from an HMM database.
\item[\emprog{hmmindex}] Index an HMM database.
\item[\emprog{hmmpfam}] Search an HMM database for matches to a query sequence.
\item[\emprog{hmmsearch}] Search a sequence database for matches to an HMM.
\end{wideitem}

\subsection{Files used in the tutorial}

The subdirectory \prog{/tutorial} in the HMMER distribution contains the
files used in the tutorial, as well as a number of examples of various
file formats that HMMER reads. The important files for the tutorial
are:

\begin{wideitem}
\item[\emprog{globins50.msf}] An alignment file of 50 aligned globin
sequences, in GCG MSF format.
\item[\emprog{globins630.fa}] A FASTA format file of 630 unaligned globin sequences.
\item[\emprog{ fn3.sto}] An alignment file of fibronectin type III
domains, in Stockholm format. (From Pfam 8.0.)
\item[\emprog{ rrm.sto}] An alignment file of RNA recognition
motif domains, in Stockholm format. (From Pfam 8.0).
\item[\emprog{ rrm.hmm}] An example HMM, built from rrm.sto
\item[\emprog{ pkinase.sto}] An alignment file of protein kinase
catalytic domains, in Stockholm format. (From Pfam 8.0).
\item[\emprog{ Artemia.fa}] A FASTA file of brine shrimp globin, which contains
nine tandemly repeated globin domains.
\item[\emprog{ 7LES\_DROME}] A SWISSPROT file of the \emph{ Drosophila} 
Sevenless sequence, a receptor tyrosine kinase with multiple domains.
\item[\emprog{ RU1A\_HUMAN}] A SWISSPROT file of the human U1A
protein sequence, which contains two RRM domains.
\end{wideitem}

Create a new directory that you can work in, and copy all the files in
\prog{tutorial} there. I'll assume for the following examples that you've
installed the HMMER programs in your path; if not, you'll need to give
a complete path name to the HMMER programs (e.g. something like \prog{
/usr/people/eddy/hmmer-2.2/binaries/hmmbuild} instead of just
\prog{hmmbuild}).

\subsection{Format of input alignment files}

HMMER starts with a multiple sequence alignment file that you
provide. HMMER can read alignments in several common formats,
including the output of the CLUSTAL family of programs, Wisconsin/GCG
MSF format, the input format for the PHYLIP phylogenetic analysis
programs, and ``alighed FASTA'' format (where the sequences in a FASTA
file contain gap symbols, so that they are all the same length).

HMMER's native alignment format is called Stockholm format, the format
of the Pfam protein database that allows extensive markup and
annotation. All these formats are documented in a later section.

The software autodetects the alignment file format, so you don't have
to worry about it.

Most of the example alignments in the tutorial are Stockholm
files. \prog{rrm.sto} is a simple example (generated by stripping all
the extra annotation off of a Pfam RNA recognition motif seed
alignment).  \prog{pkinase.sto} and \prog{fn3.sto} are original Pfam
seed alignments, with all their annotation. 

\subsection{Searching a sequence database with a single profile HMM}

One common use of HMMER is to search a sequence database for
homologues of a protein family of interest. You need a multiple
sequence alignment of the sequence family you're interested in.

\begin{srefaq}{Can I build a model from unaligned sequences?}
In principle, profile HMMs can be trained from unaligned sequences;
however, this functionality is temporarily withdrawn from HMMER.  I
recommend CLUSTALW as an excellent, freely available multiple sequence
alignment program. The original \prog{hmmt} profile HMM training
program from HMMER 1 is also still available, from
\htmladdnormallink{ftp://ftp.genetics.wustl.edu/pub/eddy/hmmer/hmmer-1.8.4.tar.Z}{ftp://ftp.genetics.wustl.edu/pub/eddy/hmmer/hmmer-1.8.4.tar.Z}.
\end{srefaq}

\subsubsection{build a profile HMM with hmmbuild}

Let's assume you have a multiple sequence alignment of a protein
domain or protein sequence family. To use HMMER to search for
additional remote homologues of the family, you want to first build a
profile HMM from the alignment. The following command builds a profile
HMM from the alignment of 50 globin sequences in \prog{ globins50.msf}:

\user{hmmbuild globin.hmm globins50.msf}

This gives the following output: 
\begin{sreoutput}
hmmbuild - build a hidden Markov model from an alignment
HMMER 2.3 (April 2003)
Copyright (C) 1992-2003 HHMI/Washington University School of Medicine
Freely distributed under the GNU General Public License (GPL)
- - - - - - - - - - - - - - - - - - - - - - - - - - - - - - - - - - - -
Alignment file:                    globins50.msf
File format:                       MSF
Search algorithm configuration:    Multiple domain (hmmls)
Model construction strategy:       MAP (gapmax hint: 0.50)
Null model used:                   (default)
Prior used:                        (default)
Sequence weighting method:         G/S/C tree weights
New HMM file:                      globin.hmm 
- - - - - - - - - - - - - - - - - - - - - - - - - - - - - - - -

Alignment:           #1
Number of sequences: 50
Number of columns:   308

Determining effective sequence number    ... done. [2]
Weighting sequences heuristically        ... done.
Constructing model architecture          ... done.
Converting counts to probabilities       ... done.
Setting model name, etc.                 ... done. [globins50]

Constructed a profile HMM (length 143)
Average score:      189.04 bits
Minimum score:      -17.62 bits
Maximum score:      234.09 bits
Std. deviation:      53.18 bits

Finalizing model configuration           ... done.
Saving model to file                     ... done.
//
\end{sreoutput}

The process takes a second or two.  \prog{hmmbuild} create a new HMM
file called \prog{globin.hmm}. This is a human and computer readable
ASCII text file, but for now you don't care. You also don't care for
now what all the stuff in the output means; I'll describe it in detail
later. The profile HMM can be treated as a compiled model of your
alignment.

\subsubsection{calibrate the profile HMM with hmmcalibrate}

This step is optional, but doing it will increase the sensitivity of
your database search.

When you search a sequence database, it is useful to get ``E-values''
(expectation values) in addition to raw scores. When you see a
database hit that scores $x$, an E-value tells you the number of hits
you would've expected to score $x$ or more just by chance in a
sequence database of this size. 

HMMER will always estimate an E-value for your hits. However, unless
you ``calibrate'' your model before a database search, HMMER uses an
analytic upper bound calculation that is extremely conservative.  An
empirical HMM calibration costs time (about 10\% the time of a
SWISSPROT search) but it only has to be done once per model, and can
greatly increase the sensitivity of a database search. To empirically
calibrate the E-value calculations for the globin model, type:

\user{hmmcalibrate globin.hmm}

which results in:
\begin{sreoutput}
hmmcalibrate -- calibrate HMM search statistics
HMMER 2.3 (April 2003)
Copyright (C) 1992-2003 HHMI/Washington University School of Medicine
Freely distributed under the GNU General Public License (GPL)
- - - - - - - - - - - - - - - - - - - - - - - - - - - - - - - - - - - -
HMM file:                 globin.hmm
Length distribution mean: 325
Length distribution s.d.: 200
Number of samples:        5000
random seed:              1051632537
histogram(s) saved to:    [not saved]
POSIX threads:            4
- - - - - - - - - - - - - - - - - - - - - - - - - - - - - - - -

HMM    : globins50
mu     :   -39.897396
lambda :     0.226086
max    :    -9.567000
//
\end{sreoutput}

This might take several minutes, depending on your machine. Go have a
cup of coffee. When it is complete, the relevant parameters are added
to the HMM file.  (Note from the ``POSIX threads: 4'' line that I'm
running on 4 CPUs on a quad-processor box. I'm impatient.)

Calibrated HMMER E-values tend to be relatively accurate. E-values of
0.1 or less are, in general, significant hits. Uncalibrated HMMER
E-values are also reliable, erring on the cautious side; uncalibrated
models may miss remote homologues.

\begin{srefaq}
{Why doesn't hmmcalibrate always give the same output, if I run it on
the same HMM?}  It's fitting a distribution to the scores obtained
from a random (Monte Carlo) simulation of a small sequence database,
and this random sequence database is different each time. You can make
\prog{hmmcalibrate} give reproducible results by making it initialize
its random number generator with the same seed, using the \prog{--seed
<x>} option, where \prog{x} is any positive integer. By default, it
chooses a ``random'' seed, which it reports in the output header. You
can reproduce an \prog{hmmcalibrate} run by passing this number as the
seed. (Trivia: the default seed is the number of seconds that have
passed since the UNIX ``epoch'' - usually January 1, 1970.
\prog{hmmcalibrate} runs started in the same second will give
identical results. Beware, if you're trying to measure the variance of
HMMER's estimated $\hat{\lambda}$ and $\hat{\mu}$ parameters...)
\end{srefaq}

\subsubsection{search the sequence database with hmmsearch}

As an example of searching for new homologues using a profile HMM,
we'll use the globin model to search for globin domains in the example
\emph{ Artemia} globin sequence in \prog{ Artemia.fa}:

\user{hmmsearch globin.hmm Artemia.fa}

The output comes in several sections, and unlike building and
calibrating the HMM, where we treated the HMM as a black box, now you
\emph{do} care about what it's saying.

The first section is the \emph{header} that tells you what program you
ran, on what, and with what options:

\begin{sreoutput}
hmmsearch - search a sequence database with a profile HMM
HMMER 2.3 (April 2003)
Copyright (C) 1992-2003 HHMI/Washington University School of Medicine
Freely distributed under the GNU General Public License (GPL)
- - - - - - - - - - - - - - - - - - - - - - - - - - - - - - - - - - - -
HMM file:                   globin.hmm [globins50]
Sequence database:          Artemia.fa
per-sequence score cutoff:  [none]
per-domain score cutoff:    [none]
per-sequence Eval cutoff:   <= 10        
per-domain Eval cutoff:     [none]
- - - - - - - - - - - - - - - - - - - - - - - - - - - - - - - -

Query HMM:   globins50
Accession:   [none]
Description: [none]
  [HMM has been calibrated; E-values are empirical estimates]
\end{sreoutput}

The second section is the \emph{ sequence top hits} list. It is a list
of ranked top hits (sorted by E-value, most significant hit first),
formatted in a BLAST-like style:

\begin{sreoutput}
Scores for complete sequences (score includes all domains):
Sequence Description                                    Score    E-value  N 
-------- -----------                                    -----    ------- ---
S13421   S13421 GLOBIN - BRINE SHRIMP                   474.3   1.7e-143   9
\end{sreoutput}

The first field is the name of the target sequence, then followed by
the description line for the sequence. The last three fields are the
raw score (in units of ``bits''), the estimated E-value, and the total
number of domains detected in the sequence.  By default, every
sequence with an E-value less than 10.0 is listed in this output.

The second section is the \emph{ domain top hits} list. By default, for
every sequence with an E-value less than 10, every domain with a raw
score greater than 0 is listed. (Read that carefully. In a later
chapter we'll discuss some caveats about how \prog{hmmsearch}
identifies domains, and how to control its output in different ways.)
Each domain detected in the search is output in a list ranked by
E-value:

\begin{sreoutput}
Parsed for domains:
Sequence Domain  seq-f seq-t    hmm-f hmm-t      score  E-value
-------- ------- ----- -----    ----- -----      -----  -------
S13421     7/9     932  1075 ..     1   143 []    76.9  7.3e-24
S13421     2/9     153   293 ..     1   143 []    63.7  6.8e-20
S13421     3/9     307   450 ..     1   143 []    59.8  9.8e-19
S13421     8/9    1089  1234 ..     1   143 []    57.6  4.5e-18
S13421     9/9    1248  1390 ..     1   143 []    52.3  1.8e-16
S13421     1/9       1   143 [.     1   143 []    51.2    4e-16
S13421     4/9     464   607 ..     1   143 []    46.7  8.6e-15
S13421     6/9     775   918 ..     1   143 []    42.2    2e-13
S13421     5/9     623   762 ..     1   143 []    23.9  6.6e-08
\end{sreoutput}

The first field is the name of the target sequence. The second field
is the number of this domain: e.g. ``6/9'' means the sixth domain of
nine total domains detected.

The fields marked ``seq-f'' and ``seq-t'' mean ``sequence from'' and
``sequence to'': the start and end points of the alignment on the
target sequence. After these two fields is a shorthand annotation for
whether the alignment is ``global'' with respect to the sequence or
not. A dot (.) means the alignment does not go all the way to the end;
a bracket ([ or ]) means it does. Thus, .. means that the alignment is
local within the sequence; [. means that the alignment starts at the
beginning of the sequence, but doesn't go all the way to its end; .]
means the alignment starts somewhere internally and goes all the way
to the end; and [] means the alignment includes the entire sequence.

Analogously, the fields marked ``hmm-f'' and ``hmm-t'' indicate the
start and end points with respect to the consensus coordinates of the
model, and the following field is a shorthand for whether the
alignment is global with respect to the \emph{model}. Here, for
instance, all the globin domains in the \emph{Artemia} sequence are
complete matches to the entire globin model -- \emph{because, by
default, \prog{hmmbuild} built the HMM to only look for those kinds of
alignments}. We'll discuss later how to modify the profile HMM for
other search styles.

The final two fields are the raw score in bits and the estimated
E-value, \emph{for the isolated domain}.  The scores for the domains
sum up to the raw score of the complete sequence.

The next section is the \emph{ alignment output}. By default, every
domain that appeared in the domain top hits list now appears as a
BLAST-like alignment. For example:

\begin{sreoutput}
Alignments of top-scoring domains:
S13421: domain 7 of 9, from 932 to 1075: score 76.9, E = 7.3e-24
                   *->eekalvksvwgkveknveevGaeaLerllvvyPetkryFpkFkdLss
                      +e a vk+ w+ v+ ++  vG  +++ l++ +P+ +++FpkF d+  
      S13421   932    REVAVVKQTWNLVKPDLMGVGMRIFKSLFEAFPAYQAVFPKFSDVPL 978  

                   adavkgsakvkahgkkVltalgdavkkldd...lkgalakLselHaqklr
                    d++++++ v +h   V t+l++ ++ ld++ +l+   ++L+e H+  lr
      S13421   979 -DKLEDTPAVGKHSISVTTKLDELIQTLDEpanLALLARQLGEDHIV-LR 1026 

                   vdpenfkllsevllvvlaeklgkeftpevqaalekllaavataLaakYk<
                   v+   fk +++vl+  l++ lg+ f+  ++ +++k+++++++ +++  + 
      S13421  1027 VNKPMFKSFGKVLVRLLENDLGQRFSSFASRSWHKAYDVIVEYIEEGLQ  1075 

                   -*
                     
      S13421     -    -    
\end{sreoutput}

The top line is the HMM consensus. The amino acid shown for the
consensus is the highest probability amino acid at that position
according to the HMM (not necessarily the highest \emph{ scoring} amino
acid, though). Capital letters mean ``highly conserved'' residues:
those with a probability of $> 0.5$ for protein models, or $> 0.9$ for
DNA models. 

The center line shows letters for ``exact'' matches to the highest
probability residue in the HMM, or a ``+'' when the match has a
positive score and is therefore considered to be ``conservative''
according to the HMM's view of \emph{ this particular position in the
model} -- not the usual definition of conservative changes in general.

The third line shows the sequence itself, of course.

\begin{srefaq}{Why does alignment output from hmmsearch or
hmmpfam include some strange almost-blank lines with -'s and *'s?}  
The consensus line includes leading and
trailing symbols \prog{*->} and \prog{<-*}, representing the
nonemitting profile HMM state paths $S \rightarrow N \rightarrow B$
and $E \rightarrow C \rightarrow T$. This little flourish makes some
sense if you know something about state paths and profile HMMs, but
contributes nothing terribly useful to the output from a user's
perspective -- except that it may confuse your output parsing scripts
when the extra symbols cause the alignment to unexpectedly wrap around
to a blank final block, as happened in this example.
\end{srefaq}

The next section of the output is the \emph{ score histogram}.  It shows
a histogram with raw score increasing along the Y axis, and the number
of sequence hits represented as a bar along the X axis. In our example
here, since there's only a single sequence, the histogram is very
boring:

\begin{sreoutput}
Histogram of all scores:
score    obs    exp  (one = represents 1 sequences)
-----    ---    ---
  474      1      0|=                                                          
\end{sreoutput}

Notice though that it's a histogram of the whole sequence hits, not
the domain hits.

You can ignore the rest of the \prog{hmmsearch} output:

\begin{sreoutput}
% Statistical details of theoretical EVD fit:
              mu =   -39.8974
          lambda =     0.2261
chi-sq statistic =     0.0000
  P(chi-square)  =          0

Total sequences searched: 1

Whole sequence top hits:
tophits_s report:
     Total hits:           1
     Satisfying E cutoff:  1
     Total memory:         16K

Domain top hits:
tophits_s report:
     Total hits:           9
     Satisfying E cutoff:  9
     Total memory:         21K
\end{sreoutput}

This is just some trailing internal info about the search that used to
be useful to me.

\subsubsection{searching major databases like NCBI NR or SWISSPROT}

HMMER reads all major database formats and does not need any special
database indexing. You can search any large sequence database you have
installed locally just by giving the full path to the database file,
e.g. something like:

\user{hmmsearch globin.hmm /nfs/databases/swiss35/sprot35.dat}

If you have BLAST installed locally, it's likely that you have a
directory (or directories) in which the BLAST databases are kept.  The
location of these directories is specified to BLAST by a shell
environment variable called \prog{BLASTDB}, which contains a
colon-delimited list of one or more directories. HMMER will read the
same environment variable. For example, if you have BLAST databases in
directories called \prog{/nfs/databases/blast-db/} and
\prog{/nfs/databases/golden-path/blast/}, and you want to search
\prog{/nfs/databases/blast-db/swissprot}, the following commands will
work (in a C shell):

\user{setenv BLASTDB /nfs/databases/blast-db/:/nfs/databases/golden-path/blast/}\\
\user{hmmsearch globin.hmm swissprot}

You'd tend to have the \prog{setenv} command as part of the local
configuration of your machine, rather than typing it at the command
line.

\subsubsection{local alignment versus global alignment}

\emph{ This is important.} HMMER does not do local (Smith/Waterman)
and global (Needleman/Wunsch) style alignments in the same way that
most computational biology analysis programs do it.  To HMMER, whether
local or global alignments are allowed is part of the \emph{model},
rather than being accomplished by running a different
\emph{algorithm}. (This will be discussed in greater detail later; it
is part of the ``Plan7'' architecture of the new HMMER2 models.)

Therefore, you need to choose what kind of alignments you want to
allow \emph{when you build the model} with \prog{hmmbuild}.  By
default, \prog{hmmbuild} builds models which allow alignments that are
global with respect to the HMM, local with respect to the sequence,
and allows multiple domains to hit per sequence. Such models will only
find complete domains.

\prog{hmmbuild} provides some standard options for common alignment
styles. The following table shows the four alignment styles supported
by \prog{ hmmbuild}, and also shows the equivalent old HMMER 1.x
search program style (to orient old-school HMMER users).

\vspace{1em}
\begin{tabular}{lllll}
Command           & w.r.t. sequence & w.r.t. HMM & multidomain & HMMER
1 equivalent \\ \hline
\prog{ hmmbuild}    & local & global & yes & hmmls \\
\prog{ hmmbuild -f} & local & local  & yes & hmmfs \\
\prog{ hmmbuild -g} & local & global & no  & hmms  \\
\prog{ hmmbuild -s} & local & local  & no  & hmmsw \\ \hline
\end{tabular}
\vspace{1em}

In brief, if you know you only want to find complete domains, use the
\prog{hmmbuild} default. If you need to find fragments (local
alignments) too, and are willing to give up some sensitivity on
complete domains to see them, use \prog{hmmbuild -f}. If you want
maximal search sensitivity, build two models and search with both of
them.

\subsection{Searching a query sequence against a profile HMM database}

A second use of HMMER is to look for known domains in a query
sequence, by searching a single sequence against a library of
HMMs. (Contrast the previous section, in which we searched a single
HMM against a sequence database.) To do this, you need a library of
profile HMMs. One such library is the PFAM database
\cite{Sonnhammer97,Bateman02}. You can also create your own.

\subsubsection{creating your own profile HMM database} 

HMM databases are just concatenated single HMM files. You can build
them either by invoking the \prog{ -A} ``append'' option of
\prog{hmmbuild}, or by concatenating HMM files you've already built.
For example, here's two ways to build an HMM database called
\prog{myhmms} that contains models of the rrm RNA recognition motif
domain, the fn3 fibronectin type III domain, and the pkinase protein
kinase catalytic domain:

\user{hmmbuild -F rrm.hmm rrm.sto}\\
\user{hmmbuild fn3.hmm fn3.sto}\\
\user{hmmbuild pkinase.hmm pkinase.sto}\\
\user{cat rrm.hmm fn3.hmm pkinase.hmm > myhmms}\\
\user{hmmcalibrate myhmms}

or:

\user{hmmbuild -A myhmms rrm.sto}\\
\user{hmmbuild -A myhmms fn3.sto}\\
\user{hmmbuild -A myhmms pkinase.sto}\\
\user{hmmcalibrate myhmms}\\

Notice that \prog{ hmmcalibrate} can be run on HMM databases as well
as single HMMs. 

Also note the \prog{-F} option on that first \prog{hmmbuild} command
line; that's the ``force'' option which allows overwriting an existing
HMM file. If you really followed instructions, you already have a file
\prog{rrm.hmm} that came with the tutorial files. HMMER will refuse to
create a new HMM if it would mean overwriting an existing one. Usually
this is just annoying, but sometimes can be a useful safety check.

\subsubsection{parsing the domain structure of a sequence with hmmpfam}

Now that you have a small HMM database called \prog{myhmms}, let's use
it to analyze the \emph{Drosophila} Sevenless sequence,
\prog{7LES\_DROME}:

\user{hmmpfam myhmms 7LES\_DROME}

Like \prog{ hmmsearch}, the \prog{ hmmpfam} output comes in several
sections. The first section is the \emph{ header}:

\begin{sreoutput}
hmmpfam - search one or more sequences against HMM database
HMMER 2.3 (April 2003)
Copyright (C) 1992-2003 HHMI/Washington University School of Medicine
Freely distributed under the GNU General Public License (GPL)
- - - - - - - - - - - - - - - - - - - - - - - - - - - - - - - - - - - -
HMM file:                 myhmms
Sequence file:            7LES_DROME
- - - - - - - - - - - - - - - - - - - - - - - - - - - - - - - -

Query sequence: 7LES_DROME
Accession:      P13368
Description:    SEVENLESS PROTEIN (EC 2.7.1.112).
\end{sreoutput}

The next section is the \emph{ sequence family classification} top hits
list, ranked by E-value. The scores and E-values here reflect the
confidence that this query sequence contains one \emph{or more} domains
belonging to a domain family. The fields have the same meaning as in
\prog{hmmsearch} output, except that the name and description are for
the HMM that's been hit.

\begin{sreoutput}
Scores for sequence family classification (score includes all domains):
Model    Description                                    Score    E-value  N 
-------- -----------                                    -----    ------- ---
pkinase  Protein kinase domain                          314.6    1.4e-94   1
fn3      Fibronectin type III domain                    176.6    4.7e-53   6
rrm      RNA recognition motif. (a.k.a. RRM, RBD, or    -40.4        1.6   1
\end{sreoutput}

The E-values for pkinase and fn3 are excellent (much less than 1), so
this sequence has domains that belong to the protein kinase and the
fibronectin type III domain families. An rrm hit is also reported
(because it is less than E=10) but at 1.6, it is not significant; we
expect to see 1.6 hits of this quality just by chance in a search of
this size. By default, like BLAST, \prog{hmmsearch} and \prog{hmmpfam}
report well down into the noise. If you want the output to be cleaner,
set an E-value threshold; for example \prog{hmmpfam -E 0.1}.

The next section is the \emph{domain parse} list, ordered by position
on the sequence (not by score). Again the fields have the same meaning
as in \prog{ hmmsearch} output:

\begin{sreoutput}
Parsed for domains:
Model    Domain  seq-f seq-t    hmm-f hmm-t      score  E-value
-------- ------- ----- -----    ----- -----      -----  -------
fn3        1/6     437   522 ..     1    84 []    48.3  2.1e-14
fn3        2/6     825   914 ..     1    84 []    13.4  1.6e-05
fn3        3/6    1292  1389 ..     1    84 []    15.9  9.4e-06
fn3        4/6    1799  1891 ..     1    84 []    63.5  5.3e-19
fn3        5/6    1899  1978 ..     1    84 []    15.2  1.1e-05
fn3        6/6    1993  2107 ..     1    84 []    20.3  3.7e-06
pkinase    1/1    2209  2483 ..     1   294 []   314.6  1.4e-94
rrm        1/1    2223  2284 ..     1    77 []   -40.4      1.6
\end{sreoutput}

Note how it's still showing us that ``rrm'' hit - \prog{7LES\_DROME}
doesn't have any RRM domains. 

The final output section is the \emph{alignment output}, just like
\prog{hmmsearch}:

\begin{sreoutput}
Alignments of top-scoring domains:
fn3: domain 1 of 6, from 437 to 522: score 48.3, E = 2.1e-14
                CS    C CCCCEEEEEECCTTCCEEEEECCC CCCCCCCEEEEE.ECCCCCC
                   *->P.saPtnltvtdvtstsltlsWsppt.gngpitgYevtyRqpkngge
                      P saP   + +++ ++ l ++W p +  ngpi+gY++++ +++ g+ 
  7LES_DROME   437    PiSAPVIEHLMGLDDSHLAVHWHPGRfTNGPIEGYRLRL-SSSEGNA 482  

                CS CCCCEEECCCCCECECCEEEEECCCCEEEEEECCC CCCC   
                   wneltvpgtttsytltgLkPgteYevrVqAvnggG.GpeS<-*
                   + e+ vp    sy+++ L++gt+Y++ +  +n +G+Gp     
  7LES_DROME   483 TSEQLVPAGRGSYIFSQLQAGTNYTLALSMINKQGeGPVA    522  
...
\end{sreoutput}

The alignment report has an extra line in this example, a ``CS''
(consensus structure) line, indicating residues expected to be in
$\alpha$-helix (H), coil (C), or $\beta$-sheet (E). (The fibronectin
type III domain is a 7-stranded $\beta$-sandwich.) This line was
picked up from the consensus secondary structure annotation
(\verb+#=GC SS_cons+) in the \prog{fn3.sto} alignment file, which many
curated Pfam alignments now have. \prog{hmmbuild} can pick up several
kinds of optional information from an annotated alignment file.

\subsubsection{obtaining the PFAM database}

The PFAM database is available from either \htmladdnormallink{\prog{
http://pfam.wustl.edu/}}{http://pfam.wustl.edu/} or
\htmladdnormallink{\prog{http://www.sanger.ac.uk/Pfam/}}{http://www.sanger.ac.uk/Pfam/}.
Download instructions are on the Web page. The PFAM HMM library is a
single large file, containing several hundred models of known protein
domains. Install it in a convenient directory and name it something
simple like \prog{Pfam}.

HMMER will look for PFAM and other files in a directory (or
directories) specified by the \prog{HMMERDB} environment variable.
For instance, if you install the PFAM HMM library as
\prog{/nfs/databases/hmmer/pfam}, the following commands will search
for domains in \prog{7LES\_DROME}:

\user{setenv HMMERDB /nfs/databases/hmmer/}\\
\user{hmmpfam pfam 7LES\_DROME}


\subsection{Creating and maintaining multiple alignments with hmmalign}

Another use of profile HMMs is to create multiple sequence alignments
of large numbers of sequences. A profile HMM can be build of a
``seed'' alignment of a small number of representative sequences, and
this profile HMM can be used to efficiently align any number of
additional sequences. 

This is how the PFAM database is updated automatically as the primary
sequence databases increase exponentially in size. The PFAM seed
alignments are curated, representative alignments that are
(relatively) stable from release to release. PFAM full alignments are
created automatically by searching a nonredundant database with the
seed model and aligning all the significant hits into a multiple
alignment using \prog{hmmalign}.

For example, to align the 630 globin sequences in \prog{globins630.fa}
to our globin model \prog{globin.hmm}, and create a new alignment file
called \prog{globins630.ali}, we'd do:

\user{hmmalign -o globins630.ali globin.hmm globins630.fa}

which would result in:

\begin{sreoutput}
hmmalign - align sequences to an HMM profile
HMMER 2.3 (April 2003)
Copyright (C) 1992-2003 HHMI/Washington University School of Medicine
Freely distributed under the GNU General Public License (GPL)
- - - - - - - - - - - - - - - - - - - - - - - - - - - - - - - - - - - -
HMM file:             globin.hmm
Sequence file:        globins630.fa
- - - - - - - - - - - - - - - - - - - - - - - - - - - - - - - -

Alignment saved in file globins630.ali
\end{sreoutput}

Using the \prog{ -o} option to specify a save file for the final
alignment is a good idea. Else, the alignment will be displayed on the
screen as output -- and an alignment of several hundred sequences will
give a fairly voluminous output.


\subsection{General notes on using the programs in HMMER}

\subsubsection{getting quick help on the command line}

If you forget the command-line syntax or available options of any of
the programs, you can type the name of the program with no other
arguments and get a short help message, including summaries of the
most common options, e.g.

\user{hmmbuild}
 
If you call any program with an option \prog{-h}, you get an augmented
help message, including version info (the software version number is
helpful if you report bugs or other problems to me) and a complete
summary of all the available options, including the
expert/experimental ones, e.g.

\user{hmmbuild -h}

Commonly used options are generally small letters, like \prog{-a}.
More infrequently used options are generally large letters, like
\prog{-A}. Expert or experimental options are generally in the GNU
long form, like \prog{--null2}.

\subsubsection{sequence file formats}

Unaligned sequence files are usually in FASTA format, but HMMER can
read many common file formats including Genbank, EMBL, and SWISS-PROT
format.

Aligned sequence files are expected to be in Stockholm format (HMMER's
native format, used by the Pfam and Rfam databases), but HMMER can
read many common alignment formats including Clustal, GCG MSF, aligned
FASTA, and Phylip format. It can also read a simple format (SELEX
format) of one line per sequence, containing the name first, followed
by the aligned sequence. Alignment files can also be used where
unaligned format files are required; the sequences will be read in one
at a time and their gaps removed.

HMMER autodetects what format its input sequence files are in. You
don't have to worry about reformatting sequences or setting options.

The autodetector is robust - so long as your file really is in one of
the formats HMMER expects. If you pass HMMER a file that is not a
sequence file, or is in an unexpected format, the autodetector may
screw up. (SELEX format, for instance, is so permissive that many
non-sequence files look like SELEX to the autodetector.) You can turn
off the autodetector and manually specify an input sequence file
format using the \prog{--informat} option to many of the programs.
This is particularly useful when driving HMMER with unattended scripts
- you can increase the robustness of your annotation pipeline by
specifying \prog{--informat fasta}, for instance, so HMMER will be
able to check that your files really are in FASTA format.

\subsubsection{using compressed files}

HMMER will automatically read sequence files compressed with gzip, but
you have to tell it what format the file is in: format autodetection
doesn't work on compressed files. For instance, if I have a compressed
FASTA file of the nonredundant NCBI database as \prog{nr.gz}, I can
search it with:

\user{hmmsearch --informat fasta my.hmm nr.gz}

HMMER simply reacts to the \prog{.gz} suffix to detect a gzip'ed file.
You must have \prog{gzip} installed and in your \prog{\$PATH} for this
to work; HMMER simply invokes \prog{gzip -dc} on the file, and starts
reading the data from the resulting pipe.

\subsubsection{reading from pipes}

HMMER can also read sequence data from standard input, through a pipe.
Again (as with gzip'ed files) format autodetection doesn't work, so
you must specify the file format manually. To read from \prog{stdin},
pass a \prog{-} as the file name; for example, if I have a script that
retrieves a bunch of FASTA records, and I want to search each
retrieved sequence against a model, I could do:

\user{cool\_sequence\_fetching\_script.pl \Verb+|+ hmmsearch --informat fasta my.hmm -}

\begin{srefaq}{Why can't HMMER autodetect sequence file formats from
gzip'ed files or pipes?}
Format autodetection looks at the file once; then it rewinds to the
beginning of the file; then it reads the data, in a second pass across
the file. When the input is from a pipe rather than a file, a UNIX
program can't assume that it can rewind a pipe. Therefore HMMER skips the
autodetection pass when reading from a pipe.  HMMER could be recoded to store the file as it
reads it in for autodetection, so it only makes a single pass over the
data -- but that's not the way it's coded now.
\end{srefaq}

\subsubsection{protein analysis versus nucleic acid analysis}

HMMER2 is only recommended for protein sequence analysis.

Nonetheless, HMMER can work with either nucleic acid models and
sequences using a nucleotide alphabet, or with protein models and
sequences using an amino acid alphabet.  HMMER will autodetect whether
your files contain RNA, DNA, or protein sequence. (An exception is
hmmpfam, which for technical reasons does not do sequence type
autodetection; to run hmmpfam on a nucleic acid query sequence, you
must specify the \prog{-n} option.)

\begin{srefaq}{Why is HMMER recommended only for protein sequence
analysis, if it can build and search with DNA models too?}  The
problem is just that I haven't systematically tested HMMER2 on DNA
sequences yet, so I don't want to recommend it. I'm a little wary
because unlike HMMER1, HMMER2 has been tuned in various ways for
Pfam-style protein alignments. \prog{hmmbuild}, for instance, by
default uses a model construction algorithm that is tuned to the
expected information content of protein alignments. The search
programs also do not expect to have to deal with chromosome-scale
target sequences, though they use memory-efficient algorithms that can
deal with large proteins.  Feel free to explore it (warily) yourself,
or you can download HMMER1 (which was used extensively for DNA
analysis) from the web site.
\end{srefaq}

Since HMMER2 is aimed at protein analysis, there isn't even any option
for searching both strands of a DNA database. If you do use HMMER2 for
DNA analysis, be forewarned that the search programs will only search
the top strand!

HMMER cannot search protein model queries against nucleic acid
sequence databases; you must translate your target database in all six
frames externally and search the resulting peptide database, if this
is what you want to do. (Alternatively, use Ewan Birney's GENEWISE
program, which can -- slowly -- search a HMMER protein model against a
nucleic acid database.)

\subsubsection{environment variables}

You can control some aspects of HMMER's behavior, particularly the
directories in which it will automatically look for sequence and HMM
databases, using shell environment variables. These are documented on
page~\pageref{section:environment}, in the section ``Shell environment
variables understood by HMMER''. The person who installed HMMER may
have set these environment variables system-wide; as a user, you can
override them in your own shell environment, for instance by adding
lines to your \prog{.cshrc}.

\subsubsection{exit status from the programs}

HMMER programs always return a normal (zero) POSIX exit status to the
shell when everything goes right. Upon any type of failure, HMMER
programs will return a nonzero exit status. (The exact nonzero number
returned upon a failure is usually uninformative -- for errors caught
cleanly by HMMER, it will simply exit with status 1. Serious crashes
and other faults caught by the operating system may return some other
nonzero code.)

If you are wrapping any kind of script (Perl, shell, whatever) around
HMMER, you can check this exit status to be sure that your calls are
succeeding.

