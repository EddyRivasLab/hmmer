\chapter{Installation}

\section{System requirements and portability}

HMMER is designed to run on UNIX platforms. The code is
POSIX-compliant ANSI C.  You need a UNIX machine and an ANSI C
compiler to build and use it.

Some optional tests and utilities use Perl. \prog{make check} will only
work if you have Perl installed. However, Perl isn't necessary to make
HMMER work.

HMMER includes support for two kinds of parallelization: POSIX threads
and PVM (Parallel Virtual Machine). Most modern UNIX OS's support
POSIX threads. HMMER's configure script attempts to automatically
detect your threads library, if you have one, and will build in
multithreading by default -- so you don't need to do anything special
to get multithreading. PVM, in contrast, is a separate, third-party
software application, and you will have to install and configure it
before building HMMER with PVM support. For this reason, PVM support
in HMMER is not enabled by default. See the PVM section of this
chapter for more details.

HMMER 2 should be easy to port to non-UNIX operating systems, provided
they support ANSI C and some reasonable level of POSIX compliance.  A
WinNT distribution is available by anonymous FTP from Time Logic,
Inc. HMMER 1 was ported by other people to Digital VAX/VMS, Apple
MacOS, Win95, and WinNT with relatively little difficulty, and I've
made efforts to improve the portability of the HMMER 2 code since
then. I would greatly appreciate receiving diffs for any ports of
HMMER to any platform.

\section{Installing a precompiled distribution}

Precompiled binary distributions of HMMER are available for at least
the following platforms:
\begin{itemize}
\item Intel PC/Linux
\item Intel PC/FreeBSD
\item Intel PC/Sun Solaris 2.x
\item Silicon Graphics/IRIX 5.x and 6.x
\item Sun Sparc/ Solaris 2.x
\item DEC Alpha/Digital UNIX 4.0
\end{itemize}

To install a precompiled distribution, do the following:

\begin{enumerate}
\item Download the distribution from \prog{http://hmmer.wustl.edu/}.

\item Uncompress and un-tar it; for example, \\
	\user{uncompress hmmer-intel-linux.tar.Z}
	\user{tar xf hmmer-intel-linux.tar}
	A new directory \prog{hmmer-xx} is created, where ``xx'' is
	the HMMER version number.
\item Go to the top level directory and run the
      configure script.\\
\user{cd hmmer-xx}
\user{./configure}
\item Edit the Makefile in the top level directory.

To run HMMER in the distribution directory, you don't need
to edit the Makefile at all. But to permanently install HMMER on your
system, set the make variables BINDIR and MANDIR to be the directories
where you want HMMER executables and man pages to be installed. If you
are installing in directories \prog{/usr/local/bin/} and
\prog{/usr/local/man/man1/},
you don't need to change anything.

\item Type \prog{make install} to install the programs and man pages. 
You may have to become root, depending on where you're installing.

\item Type \prog{make clean} to clean up.
\end{enumerate}

\section{Compiling from a source-only distribution}

\begin{enumerate}
\item Download the distribution from \prog{http://hmmer.wustl.edu/}.

\item Uncompress and un-tar it:\\
	\user{uncompress hmmer.tar.Z}
	\user{tar xf hmmer.tar}
A new directory \prog{hmmer-xx} is created, where ``xx'' is
the HMMER version number.

\item Go to the top level directory and run the
      configure script.\\
\user{cd hmmer-xx}
\user{./configure}
If you want to include the optional PVM support, do:\\
\user{./configure --with-pvm}
If you include PVM, the system that you're compiling on must already
be set up for PVM; specifically, the environment variables PVM\_ROOT
and PVM\_ARCH must be set, so HMMER can find the PVM headers and
library.

If you want to disable POSIX threads support, do:\\
\user{./configure --disable-threads}

If you want to change the choice of compilation flags (CFLAGS) or the
compiler (CC), set these as environment variables:\\
\user{env CC=gcc CFLAGS="-O6" ./configure}
For other generic configuration options, see the documentation for GNU
autoconf 2.12. -- but it may be easier to just edit the
Makefile (see below).

\item Edit the top of the Makefile. 

To build and test HMMER in its source directory, you don't need to
edit the Makefile at all. 

To permanently install HMMER on your system, set the make variables
BINDIR and MANDIR to be the directories where you want HMMER
executables and man pages to be installed. If you are installing the
programs in \prog{/usr/local/bin} and the man pages in
\prog{/usr/local/man/man1}, you don't need to change anything.

The default Makefile gets configured into a reliable but not
necessarily optimal choice of compiler and compilation flags. The
package is known to build "out of the box" on SGI IRIX, Sun Solaris,
GNU/Linux, FreeBSD, or DEC Digital Unix platforms without any special
modifications. However, you may want to change the CC or CFLAGS
variables to suit your system.  In particular, you can play with the
compiler options in CFLAGS to try to get more speed, if you're
compiler-fluent. The Makefile includes some hints for various
platforms. 

On SunOS 4.1.x systems, you will have to use the GNU gcc compiler,
because SunOS cc is not ANSI-compliant.

\item Type \prog{make} to build the programs.

\item (Optional) Type \prog{make check} to compile and run a test suite.
Some of the tests require that you have Perl installed. None of these
tests should fail.

\item Type \prog{make install} to install the programs and man pages. 
You may have to become root, depending on where you're installing.

\item Type \prog{make clean} to clean up. 
\end{enumerate}

\section{Environment variable configuration}

These steps are optional, and will only apply (I think) to
sufficiently POSIX-compliant operating systems like UNIX, Linux, or
WinNT.

HMMER reads four environment variables, and is designed to coexist
peacefully with an installation of WUBLAST or NCBI BLAST:

\begin{wideitem}
\item[\emprog{BLASTDB}] - directory location of FASTA-formatted sequence databases
\item[\emprog{BLASTMAT}] - directory location of BLAST substitution matrices
\item[\emprog{HMMERDB}] - directory location of HMM databases (e.g. PFAM)
\item[\emprog{HMMER\_NCPU}] - maximum number of CPUs to utilize in a multiprocessor
\end{wideitem}

If you have installed BLAST, you probably already have the two BLAST
environment variables set in system-wide or user-specific .cshrc
files. 

All four variables are optional. If they are set up, you can simplify
command lines to:\\
\user{hmmpfam pfam my.query}
\user{hmmsearch my.hmm swiss35}
instead of\\
\user{hmmpfam   /some/long/path/to/databases/pfam my.query}
\user{hmmsearch my.hmm /some/long/path/to/databases/swiss35}

\section{Parallelization using threads}

HMMER includes support for two kinds of parallelization: POSIX threads
and PVM (Parallel Virtual Machine). Threads support is built in by
default; PVM is not.

If you have a multi-processor UNIX machine, odds are that you have a
POSIX threads library. This kind of parallelization is easy from the
standpoint of a HMMER user. HMMER will happily use all your processors
(fewer, if you wish). The binary distributions of HMMER all come with
multithreading capability built in.

To {\em disable} threads support, add \prog{--disable-threads} to the
command line for \prog{./configure} before you compile a source
distribution. Multithreading is completely optional, and the software
will work fine without it.

Like multithreaded BLAST, a multithreaded HMMER search will use all
the available CPUs. Sometimes this is not desired; you may want HMMER
searches to leave some spare processing power for other work.  The
environment variable HMMER\_NCPU sets the maximum number of CPUs that
any HMMER job will use. You can also set a limit on an individual
process using the \prog{--cpu <n>} option.

\section{Parallelization using PVM}

Parallelization across multiple machines (as opposed to multithreading
on a single multiprocessor machine) can be done with PVM, the Parallel
Virtual Machine software from Oak Ridge National Labs.

PVM is freely available. You can obtain it from
\prog{http://www.epm.ornl.gov/}. You must install and configure PVM before
compiling PVM support into HMMER. During compilation, HMMER needs to
see the environment variables PVM\_ROOT and PVM\_ARCH, and the PVM
header files and libraries must be found in the appropriate place
under \$PVM\_ROOT.

To enable PVM support in HMMER, add \prog{--with-pvm} to the command line
for \prog{./configure} before you compile a source distribution.  PVM is
completely optional, and the software will work fine without it.

Whereas multithreading requires no special configuration once support
is compiled into HMMER, configuring a PVM cluster and using HMMER on
it is a little more involved. 

\subsection{Configuring a PVM cluster for HMMER}

Here, I will assume you're already familiar with PVM.

Designate one machine as the ``master'', and the other machines as
``slaves''. You will start your HMMER process on the master, and the
master will spawn jobs on the slaves using PVM.

Install PVM on the master and all the slaves. On the master, make sure
the environment variables PVM\_ROOT and PVM\_ARCH are set properly
(ideally, in a system-wide .cshrc file).

Add the master's name to your .rhosts or /etc/hosts.equiv file on the
slaves, so the slaves accept rsh connections from the master.

Put copies of HMMER executables in a directory on the master and all
the slaves. For each PVM-capable program (\prog{hmmcalibrate},
\prog{hmmpfam}, and \prog{hmmsearch}, there is a corresponding slave
PVM program (\prog{hmmcalibrate-pvm}, \prog{hmmpfam-pvm}, and
\prog{hmmsearch-pvm}). The master machine needs copies of all the HMMER
programs, including the slave PVM programs.  The slaves only need
copies of the three slave PVM programs. (You never need to start the
slave programs yourself; PVM does that. You just need to make sure
they're installed where PVM can see them.)

The PVM implementation of \prog{hmmpfam} needs a copy of any HMM
databases you may search to be installed on the master and every
slave. All HMM databases must be indexed with \prog{hmmindex}. The
reason is that \prog{hmmpfam} is I/O bound; the PVM implementation
can't distribute an HMM database fast enough over a typical cluster's
Ethernet. Instead, each PVM node accesses its own copy of the HMM
database, distributing the I/O load across the nodes.
\prog{hmmcalibrate} and \prog{hmmsearch}, in contrast, are
freestanding. Only the master node needs to be able to access any HMM
and/or sequence files.

Write a PVM hostfile for the cluster. Specify the location of the
HMMER executables using the ep= directive. Specify the location of
\prog{pvmd} on the slaves using the dx= directive (alternatively, you
can make sure PVM\_ROOT and PVM\_ARCH get set properly on the
slaves). For the slaves, use the wd= directive to specify the location
of the HMM databases for \prog{hmmpfam} (alternatively, you can make
sure HMMERDB gets set properly on the slaves). Use the sp= directive
to tell HMMER how many processors each node has (and hence, how many
independent PVM processes it should start); sp=1000 means 1 CPU,
sp=2000 means 2 CPUs, sp=4000 means 4 CPUS, etc.

Start the PVM by typing\\
\user{pvm hostfile} 
(where ``hostfile'' is the
name of your hostfile) on the master. Make sure all the nodes started
properly by typing 
\user{conf} 
at the PVM console prompt.
Type 
\user{quit} to exit from the PVM console,
which leaves the PVM running in the background.
You should only need to start PVM once. (We have a
PVM running continuously on our network right now, waiting
for HMMER jobs.)

Once PVM is running, at any time you can run HMMER programs on the
master and exploit your PVM, just by adding the option \prog{--pvm};
for instance,\\
\user{hmmpfam --pvm Pfam my.query} 
parallelizes a search of
a query sequence in the file my.query against the Pfam database.

Once PVM is properly configured and your slave nodes have the required
slave programs (and databases, in the case of \prog{hmmpfam}), the
only difference you will notice between the serial and the PVM version
is a (potentially massive) increase in search speed. Aside from the
addition of the \prog{--pvm} option on the command line, all other
options and input/output formats remain identical.

\subsection{Example of a PVM cluster}

The St. Louis Pfam server runs its searches using HMMER on a PVM
cluster called Wulfpack. I'll use it as a specific example of
configuring a PVM cluster. It's a little more intricate than you'd
usually need for personal use, just because of the details of running
PVM jobs in a standalone way from CGI scripts on a Web server.

The master node is the Web server, \prog{fisher}. The slave nodes are
eight rack-mounted dual processor Intel/Linux boxes called
\prog{wulf01} through \prog{wulf08}. Collectively, we refer to
this cluster as Wulfpack; it is a Beowulf-class Linux computing
cluster.

PVM 3.3.11 is installed in \prog{/usr/local/pvm3} on the master and the
slaves. 

On fisher, all HMMER executables are installed in
\prog{/usr/local/bin}.  On the wulf slave nodes, the three PVM slave
executables are installed in
\prog{/usr/local/wulfpack}.

Pfam and PfamFrag, two Pfam databases, are installed on the wulf slave
nodes in \prog{/usr/local/wulfpack.} They are converted to binary
format using \prog{hmmconvert -b}, then indexed using \prog{hmmindex}.
(Using binary format databases is a big performance win for
\prog{hmmpfam} searches, because \prog{hmmpfam} is I/O bound and  
binary HMM databases are smaller.)

An \prog{ls} of \prog{/usr/local/wulfpack} on any wulf node looks like:
\begin{verbatim}
[eddy@wulf01 /home]$ ls /usr/local/wulfpack/
Pfam             PfamFrag         hmmcalibrate-pvm   hmmsearch-pvm
Pfam.gsi         PfamFrag.gsi     hmmpfam-pvm
\end{verbatim}

The PVM hostfile for the cluster looks like:
\begin{verbatim}
# Config file for Pfam Web server PVM
#
* ep=/usr/local/bin sp=1000
fisher.wustl.edu
* lo=pfam dx=/usr/local/pvm3/lib/pvmd ep=/usr/local/wulfpack sp=2000
wulf01
wulf02
wulf03
wulf04
wulf05
wulf06
wulf07
wulf08
\end{verbatim}

Note one wrinkle specific to configuring Web servers: the web server
is running HMMER as user ``nobody'' because it's calling HMMER from a
CGI script. We can't configure a shell for ``nobody'' on the slaves,
so we create a dummy user called ``pfam'' on each wulf node.  The lo=
directive in the PVM hostfile tells the master to connect to the
slaves as user ``pfam''. On each slave, there is a user ``pfam'' with
a .rhosts that looks like:
\begin{verbatim}
   fisher nobody
   fisher.wustl.edu nobody
\end{verbatim}
which tells the wulf node to accept rsh connections from
fisher's user ``nobody''.

Also note how we use the sp= directive to tell HMMER (via PVM) that
the wulf nodes are dual processors. fisher is actually a dual
processor too, but by setting sp=1000, HMMER will only start one PVM
process on it (leaving the other CPU free to do all the things that
keep Web servers happy).

The trickiest thing is making sure PVM\_ROOT and PVM\_ARCH get set
properly.  For my own private PVM use, my \prog{.cshrc} contains the lines:
\begin{verbatim}
	setenv PVM_ROOT    /usr/local/pvm3
	setenv PVM_ARCH    `$PVM_ROOT/lib/pvmgetarch`
\end{verbatim}
But for the web server PVM, it's a little trickier. We start the Web
server PVM as user ``nobody'' on fisher using a local init script,
\prog{/etc/rc.d/init.d/pvm\_init}. With its error checking deleted for
clarity, this script basically looks like:

\begin{verbatim}
#!/bin/sh
wulfpack_conf=/home/www/pfam/pfam-3.1/wulfpack.conf
. /usr/local/pvm3/.pvmprofile
$PVM_ROOT/lib/pvmd $wulfpack_conf >/dev/null &
\end{verbatim}

We call this at boot time by adding the line
\prog{su nobody -c "sh /etc/rc.d/init.d/pvm\_init} 
to our \prog{rc.local} file. .pvmprofile is a little PVM-supplied
script that properly sets PVM\_ROOT and PVM\_ARCH, and wulfpack.conf
is our PVM hostfile.

The relevant lines of the CGI Perl script that runs HMMER jobs from
the Web server (again, heavily edited for clarity) are:

\begin{verbatim}
# Configure environment for PVM
$ENV{'HMMERDB'}    = "/usr/local/wulfpack:/home/www/pfam/data/"
$ENV{'PVM_EXPORT'} = "HMMERDB";
$output = `/usr/local/bin/hmmpfam --pvm Pfam /tmp/query`;
\end{verbatim}

The trick here is that we export the HMMERDB environment variable via
PVM, so the PVM processes on wulf nodes will know where to find their
copy of Pfam.

PVM is relatively complex, but with luck, this and the PVM
documentation give you enough information to get HMMER running on a
cluster. It's well worth it.  Wulfpack was simple to assemble; besides
the eight rack-mounted machines, there's a terminal switch, a single
console, a 10baseT Ethernet switch, and a UPS. Each machine runs stock
Red Hat Linux (we don't need no steenking Extreme Linux hype). The
whole thing cost us \$20K, but it runs HMMER searches as fast as a
16-processor SGI Origin -- and it's bigger and has more blinking
lights than an Origin, so it's more impressive to look at.

\section{Recommended systems}

HMMER is currently developed and maintained on the following three
systems:

\begin{itemize}
\item Silicon Graphics Origin200 dual R10K server, IRIX 6.5
\item NEC Versa 6050MX laptop, Intel Pentium, 
      Red Hat Linux 4.2, configured by VA/Research
\item Dell Dimension desktop, Intel PentiumII, Red Hat
      Linux 4.2
\end{itemize}

Since these are the development systems, similar systems are the least
likely to show portability problems.

At each release, HMMER is also tested on the following platforms: Sun
Sparc Solaris, Sun x86 Solaris, DEC Alpha/Digital Unix, and Intel
PC/FreeBSD. These systems are also unlikely to show portability
problems.

\section{Make targets}

There are a variety of make targets in the toplevel Makefile.  For
completeness, they are summarized here. \textit{You shouldn't need to
know this stuff.} Targets marked (Private) are targets which are not
supported in the public HMMER release. They only run in my local
development environment.

\begin{wideitem}
\item[\textbf{all}]  Compiles the source code; puts the compiled
programs in the binaries/ subdirectory.
 
\item[\textbf{check}]  Compiles and runs the testsuite.

\item[\textbf{install}] Installs the programs in BINDIR and
the man pages in MANDIR.

\item[\textbf{clean}] Cleans up the directory, leaving the
distribution files.

\item[\textbf{verify}] Runs consistency checks on the package.
(Private)

\item[\textbf{doc}] Builds the Userguide from \latex source. (Private)

\item[\textbf{dist}] Checks out a complete distribution from RCS
and assembles the .tar.Z file. (Private)

\item[\textbf{ftpdist}] Installs a new release on the FTP
site. (Private)

\item[\textbf{stable}] Symlinks a new FTP release to hmmer.tar.Z. (Private)

\item[\textbf{almostclean}] Like 'make clean', but leaves the
binaries/ subdirectory. Used for making an executable distribution.
(Private)
\end{wideitem}







