\chapter{Tutorial}

I hate reading documentation. I just want examples of how stuff works,
just enough to get me started and doing something productive. So,
here's a tutorial walk-through of some small projects with HMMER. If
you want the introduction, that's the second chapter. The tutorial
should be sufficient to get you started on work of your own. You can
read the other chapters later if you want.

\section {The programs in HMMER}

There are currently nine programs supported in the HMMER 2 package:

\begin{wideitem}
\item[\emprog{hmmalign}] Align sequences to an existing model.
\item[\emprog{hmmbuild}] Build a model from a multiple sequence alignment.
\item[\emprog{hmmcalibrate}] Takes an HMM and empirically determines
parameters that are used to make searches more sensitive, by
calculating more accurate expectation value scores (E-values).
\item[\emprog{hmmconvert}] Convert a model file into different formats,
including a compact HMMER 2 binary format, and ``best effort''
emulation of GCG profiles.
\item[\emprog{hmmemit}] Emit sequences probabilistically from a profile HMM.
\item[\emprog{hmmfetch}] Get a single model from an HMM database.
\item[\emprog{hmmindex}] Index an HMM database.
\item[\emprog{hmmpfam}] Search an HMM database for matches to a query sequence.
\item[\emprog{hmmsearch}] Search a sequence database for matches to an HMM.
\end{wideitem}

HMMER also provides a number of utility programs which are not HMM
programs, but may be useful:

\begin{wideitem}
\item[\emprog{alistat}] Show some simple statistics about a sequence
alignment file.
\item[\emprog{sfetch}] Retrieve a (sub-)sequence from a sequence file.
\item[\emprog{seqstat}] Show some simple statistics about a sequence file.
\item[\emprog{sreformat}] Reformat a sequence file into a different format.
\end{wideitem}

The following four programs are not included in the current release,
but are planned (i.e. they're vaporware, but it may be useful to you
to know that they're on the drawing board:)

\begin{wideitem}
\item[\emprog{hmmconfig}] Alter the search configuration of an existing model.
\item[\emprog{hmminfo}] Display summary information about the model(s) in a file.
\item[\emprog{hmmtrain}] Train a model from initially unaligned sequences,
producing both a model and a multiple alignment.
\item[\emprog{hmmview}] Graphical viewer and editor for HMMs.
\end{wideitem}

\section{Files used in the tutorial}

The subdirectory \prog{/Demos} in the HMMER distribution contains the
files used in the tutorial, as well as a number of examples of various
file formats that HMMER reads. The important files for the tutorial
are:

\begin{wideitem}
\item[\emprog{globins50.msf}] An MSF format alignment file of 50 aligned globin sequences.
\item[\emprog{ globins630.fa}] A FASTA format file of 630 unaligned globin sequences.
\item[\emprog{ fn3.slx}] A SELEX format alignment file of fibronectin type III domains.
\item[\emprog{ rrm.slx}] A SELEX format alignment file of RNA recognition
motif domains.
\item[\emprog{ pkinase.slx}] A SELEX format alignment file of protein kinase
catalytic domains.
\item[\emprog{ Artemia.fa}] A FASTA file of brine shrimp globin, which contains
nine tandemly repeated globin domains.
\item[\emprog{ 7LES\_DROME}] A SWISSPROT file of the {\em Drosophila} 
Sevenless sequence, a receptor tyrosine kinase with multiple domains.
\end{wideitem}

Create a new directory that you can work in, and copy all the files in
\prog{Demos} there. I'll assume for the following examples that you've
installed the HMMER programs in your path; if not, you'll need to give
a complete path name to the HMMER programs (e.g. something like \prog{
/usr/people/eddy/ hmmer-2.0/hmmbuild} instead of just \prog{hmmbuild}).

\section{Searching a sequence database with a single profile HMM}

One common use of HMMER is to search a sequence database for
homologues of a protein family of interest. You need a multiple
sequence alignment of the sequence family you're interested in.
(Profile HMMs can be trained from unaligned sequences; however, this
functionality is temporarily withdrawn from HMMER. I recommend
CLUSTALW as an excellent, freely available multiple sequence alignment
program.)

\subsection{HMM construction with \prog{ hmmbuild}}

Let's assume you have a multiple sequence alignment of a protein
domain or protein sequence family. To use HMMER to search for
additional remote homologues of the family, you want to first build a
profile HMM from the alignment. The following command builds a profile
HMM from the alignment of 50 globin sequences in \prog{ globins50.msf}:

\vspace{1.5em}
\user{hmmbuild globin.hmm globins50.msf}
\vspace{-1.5em}
{\small\begin{verbatim}
hmmbuild - build a hidden Markov model from an alignment
HMMER 2.0 (June 1998)
Copyright (C) 1992-1998 Washington University School of Medicine
HMMER is freely distributed under the GNU General Public License (GPL).
- - - - - - - - - - - - - - - - - - - - - - - - - - - - - - - - - - - -
Training alignment:                globins50.msf
Number of sequences:               50
Search algorithm configuration:    Multiple domain (hmmls)
Model construction strategy:       MAP (gapmax hint: 0.50)
Prior used:                        (default)
Prior strategy:                    Dirichlet
Sequence weighting method:         G/S/C tree weights
- - - - - - - - - - - - - - - - - - - - - - - - - - - - - - - -
Determining effective sequence number    ... done. [13]
Weighting sequences heuristically        ... done.
Constructing model architecture          ... done.
Converting counts to probabilities       ... done.
Setting model name, etc.                 ... done. [globins50]

Constructed a profile HMM (length 162)
Average score:      283.03 bits
Minimum score:      137.32 bits
Maximum score:      343.50 bits
Std. deviation:      53.21 bits

Finalizing model configuration           ... done.
Saving model to file                     ... done. [globin.hmm]
\end{verbatim}}

The process takes a second or two.  \prog{ hmmbuild} create a new HMM
file called \prog{ globin.hmm}. This is a human and computer readable
ASCII text file, but for now you don't care. You also don't care for
now what all the stuff in the output means; I'll describe it in detail
later. The profile HMM can be treated as a compiled model of your
alignment.

\subsection{HMM calibration with \prog{ hmmcalibrate}}

This step is optional, but doing it will increase the sensitivity of
your database search.

When you search a sequence database, it is useful to get ``E-values''
(expectation values) in addition to raw scores. When you see a
database hit that scores $x$, an E-value tells you the number of hits
you would've expected to score $x$ or more just by chance in a
sequence database of this size. 

HMMER will always estimate an E-value for your hits. However, unless
you ``calibrate'' your model before a database search, HMMER uses an
analytic upper bound calculation that is extremely conservative.  An
empirical HMM calibration costs time (about 10\% the time of a
SWISSPROT search) but it only has to be done once per model, and can
greatly increase the sensitivity of a database search. To empirically
calibrate the E-value calculations for the globin model, type:

\vspace{1.5em}
\user{hmmcalibrate globin.hmm}
\vspace{-1.5em}
{\small\begin{verbatim}
hmmcalibrate -- calibrate HMM search statistics
HMMER 2.0 (June 1998)
Copyright (C) 1992-1998 Washington University School of Medicine
HMMER is freely distributed under the GNU General Public License (GPL).
- - - - - - - - - - - - - - - - - - - - - - - - - - - - - - - - - - - -
HMM file:                 globin.hmm
Length distribution mean: 325
Length distribution s.d.: 200
Number of samples:        5000
random seed:              895790561
histogram(s) saved to:    [not saved]
- - - - - - - - - - - - - - - - - - - - - - - - - - - - - - - -

HMM    : globins50
mu     :  -107.928612
lambda :     0.186579
max    :   -63.859001
//
\end{verbatim}}

This takes several minutes. Go have a cup of coffee. When it is
complete, the relevant parameters are added to the HMM file.

Calibrated HMMER E-values tend to be relatively accurate. E-values of
0.1 or less are, in general, very significant hits. Uncalibrated HMMER
E-values are also reliable, erring on the cautious side; uncalibrated
models may miss remote homologues.

\subsection{Sequence database search with \prog{ hmmsearch}}

As an example of searching for new homologues using a profile HMM,
we'll use the globin model to search for globin domains in the example
{\em Artemia} globin sequence in \prog{ Artemia.fa}:

\vspace{1.5em}
\user{hmmsearch globin.hmm Artemia.fa}

The output comes in several sections, and unlike building and
calibrating the HMM (where we treated the HMM as a black box), now you
{\em do} care about what it's saying.

The first section is the {\em header} that tells you waht program you
ran, on what, and with what options:

{\small\begin{verbatim}
hmmsearch - search a sequence database with a profile HMM
HMMER 2.0 (June 1998)
Copyright (C) 1992-1998 Washington University School of Medicine
HMMER is freely distributed under the GNU General Public License (GPL).
- - - - - - - - - - - - - - - - - - - - - - - - - - - - - - - - - - - -
HMM file:                 globin.hmm [globins50]
Sequence database:        Artemia.fa
- - - - - - - - - - - - - - - - - - - - - - - - - - - - - - - -

Query HMM:  globins50  
  [HMM has been calibrated; E-values are empirical estimates]
\end{verbatim}}

The second section is the {\em sequence top hits} list. It is a list
of ranked top hits (sorted by E-value, most significant hit first),
formatted in a BLAST-like style:

{\small\begin{verbatim}
Scores for complete sequences (score includes all domains):
Sequence Description                                    Score    E-value  N 
-------- -----------                                    -----    ------- ---
S13421   S13421 GLOBIN - BRINE SHRIMP                   335.5     1e-101   8
\end{verbatim}}

The first field is the name of the target sequence, then followed by
the description line for the sequence. The last three fields are the
raw score (in units of ``bits''), the estimated E-value, and the total
number of domains detected in the sequence.  By default, every
sequence with an E-value over 10.0 is listed in this output.

The second section is the {\em domain top hits} list. By default, for
every sequence with an E-value less than 10, every domain with a
non-zero raw score is listed. (Read that carefully. In a later chapter
we'll discuss some caveats about how \prog{ hmmsearch} identifies
domains, and how to control its output in different ways.) Each domain
detected in the search is output in a list ranked by E-value:

{\small\begin{verbatim}
Parsed for domains:
Sequence Domain  seq-f seq-t    hmm-f hmm-t      score  E-value
-------- ------- ----- -----    ----- -----      -----  -------
S13421     6/8     928  1075 ..     1   162 []    65.8  1.5e-20
S13421     2/8     149   288 ..     1   162 []    59.4  1.3e-18
S13421     3/8     303   450 ..     1   162 []    58.4  2.6e-18
S13421     8/8    1238  1390 ..     1   162 []    42.5  1.6e-13
S13421     5/8     771   918 ..     1   162 []    33.8  3.3e-12
S13421     7/8    1085  1234 ..     1   162 []    32.0  4.6e-12
S13421     4/8     454   607 ..     1   162 []    26.1  1.4e-11
S13421     1/8       1   139 [.     1   162 []    21.9    3e-11
\end{verbatim}}

The first field is the name of the target sequence. The second field
is the number of this domain: e.g. ``6/8'' means the sixth domain of
eight total domains detected. 

The fields marked ``seq-f'' and ``seq-t'' mean ``sequence from'' and
``sequence to'': the start and end points of the alignment on the
target sequence. After these two fields is a shorthand annotation for
whether the alignment is ``global'' with respect to the sequence or
not. A dot (.) means the alignment does not go all the way to the end;
a bracket ([ or ]) means it does. Thus, .. means that the alignment is
local within the sequence; [. means that the alignment starts at the
beginning of the sequence, but doesn't go all the way to its end; .]
means the alignment starts somewhere internally and goes all the way
to the end; and [] means the alignment includes the entire sequence.

Analogously, the fields marked ``hmm-f'' and ``hmm-t'' indicate the
start and end points with respect to the consensus coordinates of the
model, and the following field is a shorthand for whether the
alignment is global with respect to the {\em model}. Here, for
instance, all the globin domains in the {\em Artemia} sequence are
complete matches to the entire globin model -- {\em because, by
default, \prog{ hmmbuild} built the HMM to only look for those kinds of
alignments}. We'll discuss later how to modify the profile HMM for
other search styles.

The final two fields are the raw score in bits and the estimated
E-value, {\em for the isolated domain}. Because of the method HMMER
uses to correct raw scores for biased sequence composition, the raw
scores for the domains do not necessarily sum up to the raw score of
the sequence.

The next section is the {\em alignment output}. By default, every
domain that appeared in the domain top hits list now appears as a
BLAST-like alignment. For example:

{\small\begin{verbatim}
Alignments of top-scoring domains:
S13421: domain 6 of 8, from 928 to 1075: score 65.8, E = 1.5e-20
                   *->vilealvnssShLSaeekalVkslWYgKVegnaeeiGaeaLgRlFvv
                      +           LSa e a Vk++W   V+ ++ ++G  ++  lF +
      S13421   928    G-----------LSAREVAVVKQTW-NLVKPDLMGVGMRIFKSLFEA 962  

                   YPwTqryFphFgdLssldavkgspkvKaHGkKVltalgdavkhLDdtgnl
                   +P  q+ Fp+F+d+ +ld +++ p v +H   V t l++ ++ LD   nl
      S13421   963 FPAYQAVFPKFSDV-PLDKLEDTPAVGKHSISVTTKLDELIQTLDEPANL 1011 

                   kgalakLSelHadklrVDPeNFklLghvlvvvLaehfgkdftPevqAAwd
                   +    +L+e H   lrV+   Fk +g+vlv  L   +g  f+  +  +w 
      S13421  1012 ALLARQLGEDH-IVLRVNKPMFKSFGKVLVRLLENDLGQRFSSFASRSWH 1060 

                   KflagvanaLahKYr<-*
                   K++++++  +++      
      S13421  1061 KAYDVIVEYIEEGLQ    1075 
\end{verbatim}}

The top line is the HMM consensus. The amino acid shown for the
consensus is the highest probability amino acid at that position
according to the HMM (not necessarily the highest {\em scoring} amino
acid, though). Capital letters mean ``highly conserved'' residues:
those with a probability of $> 0.5$ for protein models, or $> 0.9$ for
DNA models. 

The center line shows letters for ``exact'' matches to the highest
probability residue in the HMM, or a ``+'' when the match has a
positive score and is therefore considered to be ``conservative''
according to the HMM's view of {\em this particular position in the
model} -- not the usual definition of conservative changes in general.

The third line shows the sequence itself, of course.

The next section of the output is the {\em score histogram}.  It shows
a histogram with raw score increasing along the Y axis, and the number
of sequence hits represented as a bar along the X axis. In our example
here, since there's only a single sequence, the histogram is very
boring:

{\small\begin{verbatim}
Histogram of all scores:
score    obs    exp  (one = represents 1 sequences)
-----    ---    ---
  335      1      0|=                                                          
\end{verbatim}}

Notice though that it's a histogram of the whole sequence hits, not
the domain hits.

You can ignore the rest of the \prog{ hmmsearch} output:

{\small\begin{verbatim}
% Statistical details of theoretical EVD fit:
              mu =  -107.9286
          lambda =     0.1866
chi-sq statistic =     0.0000
  P(chi-square)  =          0

Whole sequence top hits:
tophits_s report:
     Total hits:           1
     Satisfying E cutoff:  1
     Total memory:         15K

Domain top hits:
tophits_s report:
     Total hits:           8
     Satisfying E cutoff:  8
     Total memory:         20K
\end{verbatim}}

This is just some trailing internal info about the search that's
useful to me sometimes, but probably not to you.

\subsection{Searching major databases like NR or SWISSPROT}

HMMER reads all major database formats and does not need any special
database indexing. You can search any large sequence database you have
installed locally just by giving the full path to the database file,
e.g. something like:

\vspace{1.5em}
\user{hmmsearch globin.hmm /nfs/databases/swiss35/sprot35.dat}

If you have BLAST installed locally, it's likely that you have a
directory (or directories) in which the BLAST databases are kept.
These directories are specified in an environment variable called \prog{
BLASTDB}. HMMER will read the same environment variable. For example,
if you have a BLAST database called \prog{
/nfs/databases/blast-db/swiss35}, the following commands will work:

\vspace{1.5em}
\user{setenv BLASTDB /nfs/databases/blast-db/}
\user{hmmsearch globin.hmm swiss35}

Obviously, you'd tend to have the \prog{ setenv} command as part of the
local configuration of your machine, rather than typing it at the
command line.

\subsection{Local alignment searches with \prog{ hmmsearch}}

{\em This is extremely important.} HMMER does not do local
(Smith/Waterman) and global (Needleman/Wunsch) style alignments in the
same way that most computational biology analysis programs do it.  To
HMMER, whether local or global alignments are allowed is part of the
{\em model}, rather than being accomplished by running a different
{\em algorithm}. (This will be discussed in greater detail later; it
is part of the ``Plan7'' architecture of the new HMMER2 models.)

Therefore, you need to choose what kind of alignments you want to
allow {\em when you build the model} with \prog{ hmmbuild}.  By default,
\prog{ hmmbuild} builds models which allow alignments that are global
with respect to the HMM, local with respect to the sequence, and
allows multiple domains to hit per sequence. Such models will only
find complete domains. 

\prog{ hmmbuild} provides some standard options for common alignment
styles. The following table shows the four alignment styles supported
by \prog{ hmmbuild}, and also shows the equivalent old HMMER 1.x search
program style (to orient experienced HMMER users).

\vspace{1em}
\begin{tabular}{lllll}
Command           & w.r.t. sequence & w.r.t. HMM & multidomain & HMMER
1 equivalent \\ \hline
\prog{ hmmbuild}    & local & global & yes & hmmls \\
\prog{ hmmbuild -f} & local & local  & yes & hmmfs \\
\prog{ hmmbuild -g} & local & global & no  & hmms  \\
\prog{ hmmbuild -s} & local & local  & no  & hmmsw \\ \hline
\end{tabular}
\vspace{1em}

In brief, if you want maximum sensitivity at the expense of only
finding complete domains, use the \prog{ hmmbuild} default. If you need
to find fragments (local alignments) too, and are willing to give up
some sensitivity to see them, use \prog{ hmmbuild -f}. If you want the
best of both worlds, build two models and search with both of them.

\section{Searching a query sequence against a profile HMM database}

A second use of HMMER is to look for known domains in a query
sequence, by searching a single sequence against a library of
HMMs. (Contrast the previous section, in which we searched a single
HMM against a sequence database.) To do this, you need a library of
profile HMMs. One such library is our PFAM database
\cite{Sonnhammer97,Sonnhammer98}, and you can also create your own.

\subsection{Creating your own profile HMM database} 

HMM databases are simply concatenated single HMM files. You can build
them either by invoking the \prog{ -A} ``append'' option of \prog{
hmmbuild}, or by concatenating HMM files you've already built.  For
example, here's two ways to build an HMM database called \prog{ myhmms}
that contains models of the rrm RNA recognition motif domain, the fn3
fibronectin type III domain, and the pkinase protein kinase catalytic
domain:

\vspace{1.5em}
\user{hmmbuild rrm.hmm rrm.slx}
\user{hmmbuild fn3.hmm fn3.slx}
\user{hmmbuild pkinase.hmm pkinase.slx}
\user{cat rrm.hmm fn3.hmm pkinase.hmm > myhmms}
\user{hmmcalibrate myhmms}

or:

\vspace{1.5em}
\user{hmmbuild -A myhmms rrm.slx }
\user{hmmbuild -A myhmms fn3.slx }
\user{hmmbuild -A myhmms pkinase.slx }
\user{hmmcalibrate myhmms }

Notice that \prog{ hmmcalibrate} can be run on HMM databases as well as
single HMMs.

\subsection{Parsing the domain structure of a sequence with \prog{ hmmpfam}}

Now that you have a small HMM database called \prog{ myhmms}, let's use
it to analyze the {\em Drosophila} Sevenless sequence, \prog{
7LES\_DROME}:

\vspace{1.5em}
\user{hmmpfam myhmms 7LES\_DROME }

Like \prog{ hmmsearch}, the \prog{ hmmpfam} output comes in several
sections. The first section is the {\em header}:

{\small\begin{verbatim}
hmmpfam - search a single seq against HMM database
HMMER 2.0 (June 1998)
Copyright (C) 1992-1998 Washington University School of Medicine
HMMER is freely distributed under the GNU General Public License (GPL).
- - - - - - - - - - - - - - - - - - - - - - - - - - - - - - - - - - - -
HMM file:                 myhmms
Sequence file:            7LES_DROME
- - - - - - - - - - - - - - - - - - - - - - - - - - - - - - - -
Query:  7LES_DROME  SEVENLESS PROTEIN (EC 2.7.1.112).
\end{verbatim}}

The next section is the {\em sequence family classification} top hits
list, ranked by E-value. The scores and E-values here reflect the
confidence that this query sequence contains one {\em or more} domains
belonging to a domain family. The fields have the same meaning as in
\prog{ hmmsearch} output, except that the name and description are for
the HMM that's been hit.

{\small\begin{verbatim}
Scores for sequence family classification (score includes all domains):
Sequence Description                                    Score    E-value  N 
-------- -----------                                    -----    ------- ---
pkinase                                                 303.3      3e-87   1
fn3                                                     171.8    1.1e-47   6
\end{verbatim}}

The next section is the {\em domain parse} list, ordered by position
on the sequence (not by score). Again the fields have the same meaning
as in \prog{ hmmsearch} output:

{\small\begin{verbatim}
Parsed for domains:
Sequence Domain  seq-f seq-t    hmm-f hmm-t      score  E-value
-------- ------- ----- -----    ----- -----      -----  -------
fn3        1/6     437   522 ..     1    84 []    48.0  2.1e-10
fn3        2/6     825   914 ..     1    84 []    12.6     0.21
fn3        3/6    1292  1389 ..     1    84 []    15.2     0.13
fn3        4/6    1799  1891 ..     1    84 []    62.4  9.4e-15
fn3        5/6    1899  1978 ..     1    84 []    13.7     0.17
fn3        6/6    1993  2107 ..     1    84 []    18.4    0.067
pkinase    1/1    2209  2483 ..     1   278 []   303.3    3e-87
\end{verbatim}}

The final output section is the {\em alignment output}, just like \prog{
hmmsearch}:

{\small\begin{verbatim}
Alignments of top-scoring domains:
fn3: domain 1 of 6, from 437 to 522: score 48.0, E = 2.1e-10
                   *->P.saPtnltvtdvtstsltlsWsppt.gngpitgYevtyRqpkngge
                      P saP   + +++ ++ l ++W p +  ngpi+gY+++  ++++g+ 
  7LES_DROME   437    PiSAPVIEHLMGLDDSHLAVHWHPGRfTNGPIEGYRLRL-SSSEGNA 482  

                   wneltvpgtttsytltgLkPgteYtvrVqAvnggG.GpeS<-*
                   + e+ vp+   sy+++ L++gt+Yt+ +  +n +G+Gp     
  7LES_DROME   483 TSEQLVPAGRGSYIFSQLQAGTNYTLALSMINKQGeGPVA    522  
...
\end{verbatim}}

\subsection{Downloading the PFAM database}

The PFAM database is available from either \htmladdnormallink{\prog{
http://pfam.wustl.edu/}}{http://pfam.wustl.edu/} or \linebreak 
\htmladdnormallink{\prog{http://www.sanger.ac.uk/Pfam/}}{http://www.sanger.ac.uk/Pfam/}.
Download instructions are on the Web page. The PFAM HMM library is a
single large file, containing several hundred models of known protein
domains. Install it in a convenient directory and name it something
simple like \prog{ pfam}.

HMMER will look for PFAM and other files in a directory (or
directories) specified by the \prog{ HMMERDB} environment variable.  For
instance, if you install the PFAM HMM library as \linebreak \prog{
/nfs/databases/hmmer/pfam}, the following commands will search 
for domains in \prog{ 7LES\_DROME}:

\vspace{1.5em}
\user{setenv HMMERDB /nfs/databases/hmmer/}
\user{hmmpfam pfam 7LES\_DROME}

\section{Maintaining multiple alignments with \prog{ hmmalign}}

Another use of profile HMMs is to create multiple sequence alignments
of large numbers of sequences. A profile HMM can be build of a
``seed'' alignment of a small number of representative sequences, and
this profile HMM can be used to efficiently align any number of
additional sequences. 

This is in fact how the PFAM database is updated as the main SPTREMBL
database increases in size. The PFAM seed alignments are (relatively)
stable from release to release; PFAM full alignments are created
automatically by searching SPTREMBL with the seed model and aligning
all the significant hits into a multiple alignment using \prog{
hmmalign}.

For example, to align the 630 globin sequences in \prog{ globins630.fa}
to our globin model \prog{ globin.hmm}, and create a new alignment file
called \prog{ globins630.ali}, we'd do:

\vspace{1.5em}
\user{hmmalign -o globins630.ali globin.hmm globins630.fa}
\vspace{-1.5em}
{\small\begin{verbatim}
hmmalign - align sequences to an HMM profile
HMMER 2.0 (June 1998)
Copyright (C) 1992-1998 Washington University School of Medicine
HMMER is freely distributed under the GNU General Public License (GPL).
- - - - - - - - - - - - - - - - - - - - - - - - - - - - - - - - - - - -
HMM file:             globin.hmm
Sequence file:        globins630.fa
- - - - - - - - - - - - - - - - - - - - - - - - - - - - - - - -

Alignment saved in file globins630.ali
\end{verbatim}}

Using the \prog{ -o} option to specify a save file for the final
alignment is a good idea; else, the alignment will be displayed on the
screen as output (and an alignment of several hundred sequences will
give a fairly voluminous output).


