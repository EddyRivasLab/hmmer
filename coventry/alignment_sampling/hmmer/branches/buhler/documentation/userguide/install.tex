\section{Installation}
\label{section:installation}

\subsection{Quick installation instructions}

\subsubsection{configuring, compiling, and installing a source code distribution}

Download the source tarball (\prog{hmmer.tar.gz}) from 
\htmladdnormallink{ftp://ftp.genetics.wustl.edu/pub/eddy/hmmer/}
                  {ftp://ftp.genetics.wustl.edu/pub/eddy/hmmer/}
or \\
\htmladdnormallink{http://hmmer.wustl.edu/}
                  {http://hmmer.wustl.edu/}.

Unpack the software:

\user{tar xvf hmmer.tar.gz}

Go into the newly created top-level directory (named \prog{hmmer-xx},
where \prog{xx} is a release number):

\user{cd hmmer-@RELEASE@}

Configure for your system, and build the programs:

\user{./configure}\\
\user{make}

Run the automated testsuite. This is optional.  All these tests should
pass:

\user{make check}

The programs are now in the \prog{src/} subdirectory. The man pages
are in the \prog{documentation/man} subdirectory. You can manually
move or copy all of these to appropriate locations if you want. You
will want the programs to be in your \$PATH.

Optionally, you can install the man pages and programs in system-wide
directories. If you are happy with the default (programs in
\prog{/usr/local/bin/} and man pages in \prog{/usr/local/man/man1}),
do:

\user{make install}

(You might need to be root when you install, depending on the
permissions on your /usr/local directories.)

That's all.  Each of these steps is documented in more detail below,
including how to change the default installation directories for
\prog{make install}.

\subsubsection{configuring and installing a precompiled binary distribution}

Alternatively, you can obtain a precompiled binary distribution of
HMMER from \htmladdnormallink{http://hmmer.wustl.edu/}
{http://hmmer.wustl.edu/}. Thanks to generous hardware support from
many manufacturers, binary distributions are available for most common
UNIX and UNIX-like OS's. For example, the distribution for Intel
x86/GNU Linux machines is
\prog{hmmer-@RELEASE@.bin.intel-linux.tar.gz}.

After you download a binary distribution, unpack it:

\user{tar xvf hmmer.bin.intel-linux.tar.gz}

HMMER is now in the newly created top-level directory (named
\prog{hmmer-xx}, where \prog{xx} is a release number). Go into
it:

\user{cd hmmer-@RELEASE@}

You don't really need to do anything else. The programs are in the
\prog{binaries/} subdirectory. The man pages are in the
\prog{documentation/man} subdirectory.  The PDF copy of the Userguide
is in the top level HMMER directory (Userguide.pdf).  You can manually
move or copy all of these to appropriate locations if you want. You
will want the programs to be in your \$PATH.

However, you'll often want to install in a more permanent place.  To
configure with the default locations (programs in
\prog{/usr/local/bin/} and man pages in \prog{/usr/local/man/man1})
and install everything, do:

\user{./configure}\\
\user{make install}

If you want to install in different places than the defaults, keep
reading; see the beginning of the section on running the
\prog{configure} script.

\subsection{System requirements and portability}

HMMER is designed to run on UNIX platforms. The code is
POSIX-compliant ANSI C.  You need a UNIX operating system to run it.
You need an ANSI C compiler if you want to build it from source. 

Linux and Apple Macintosh OS/X both count as UNIX. Microsoft operating
systems do not. However, HMMER is known to be easy to port to
Microsoft Windows and other non-UNIX operating systems, provided that
the platform supports ANSI C and some reasonable level of POSIX
compliance.

Running the testsuite (\prog{make check}) requires that you have Perl
(specifically, \verb+/usr/bin/perl+). However, Perl isn't necessary to
make HMMER work.

HMMER has support for two kinds of parallelization: POSIX
multithreading and PVM (Parallel Virtual Machine) clustering.  Both
are optional, not compiled by default; they are enabled by passing the
\prog{--enable-threads} or \prog{--enable-pvm} options to the
\prog{./configure} script before compilation. The precompiled binary
distributions generally support multithreading but not PVM.

\subsection{Running the configure script}

The GNU \prog{configure} script that comes with HMMER has a number of
options. You can see them all by doing:

\user{./configure --help}

Most customizations of HMMER are done at the \prog{./configure}
command line.

\subsubsection{setting installation targets}

The most important \prog{configure} options set the installation
directories for \prog{make install} to be appropriate to your system.
What you need to know is that HMMER installs only two types of files:
the programs, and man pages. It installs the programs in
\prog{--bindir} (which defaults to \prog{/usr/local/bin}), and the man
pages in the \prog{man1} subdirectory of \prog{--mandir} (default
\prog{/usr/local/man}). So, say you want \prog{make install} to
instead install programs in \prog{/usr/bioprogs/bin/} and man pages in
\prog{/usr/share/man/man1}; you could configure with:

\user{./configure --mandir=/usr/share/man --bindir=/usr/bioprogs/bin}

That's really all you need to know, since HMMER installs so few
different types of files. But just so you know; a GNU \prog{configure}
script is very flexible, and it has shortcuts that accomodates several
standard conventions for where programs get installed. 

One common strategy is to install all files under one directory, like
the default \prog{/usr/local}. To change this prefix to something
else, say \prog{/usr/mylocal/} (so that programs go in
\prog{/usr/mylocal/bin} and man pages in \prog{/usr/mylocal/man/man1},
you can use the \prog{--prefix} option:

\user{./configure --prefix=/usr/mylocal}

Another common strategy (especially in multiplatform environments) is
to put programs in an architecture-specific directory like
\prog{/usr/share/Linux/bin} while keeping man pages in a shared,
architecture-independent directory like \prog{/usr/share/man/man1}.
GNU configure uses \prog{--exec-prefix} to set the path to
architecture dependent files. This normally defaults to being the same
as \prog{--prefix}, but you could change it, for example, by:

\user{./configure --prefix=/usr/share --exec-prefix=/usr/share/Linux/}\\

In summary, a complete list of the \prog{./configure} installation
options that affect HMMER:

\begin{tabular}{lll}
Option                       &   Meaning                       & Default\\ \hline
\prog{--prefix=PREFIX}       & architecture independent files  & \prog{/usr/local/} \\
\prog{--exec-prefix=EPREFIX} & architecture dependent files    & PREFIX\\
\prog{--bindir=DIR}          & programs                        & EPREFIX/bin/\\
\prog{--mandir=DIR}          & man pages                       & PREFIX/man/\\
\end{tabular}

These are the only \prog{configure} options that work for both the
full source distribution of HMMER, and for the binary
distributions. The remaining options only affect the configuration of
the source distribution; they have no effect on a precompiled
binary-only distribution.


\subsubsection{setting compiler and compiler flags}

By default, \prog{configure} searches first for the GNU C compiler
\prog{gcc}, and if that is not found, for a compiler called \prog{cc}. 
This can be overridden by specifying your compiler with the \prog{CC}
environment variable.

By default, the compiler's optimization flags are set to
\prog{-g -O2} for \prog{gcc}, or \prog{-g} for other compilers.
This can be overridden by specifying optimization flags with the
\prog{CFLAGS} environment variable. 

For example, to use an Intel C compiler in
\prog{/usr/intel/ia32/bin/icc} with 
optimization flags \prog{-O3 -ip}, you could do:

\user{env CC=/usr/intel/ia32/bin/icc CFLAGS="-O3 -ip" ./configure}

which is a one-line shorthand that does the same thing as the C-shell
commands:

\user{setenv CC     /usr/intel/ia32/bin/icc}\\
\user{setenv CFLAGS "-O3 -ip"}\\
\user{./configure}

If you are using a non-GNU compiler, you almost certainly want to set
\prog{CFLAGS} to some sensible optimization flags for your platform
and compiler. The \prog{-g} default generated unoptimized
code. \emph{At a minimum, turn on your compiler's default
optimizations with \prog{CFLAGS=-O}.} Otherwise, HMMER will run much
slower than it should. If speed is crucial to you, it is often worth a
little playing around with different compiler optimization options.

\subsubsection{turning on POSIX thread support for multiprocessors}

HMMER can run in parallel on multiple processors. To enable
this, use the \prog{--enable-threads} option:

\user{./configure --enable-threads}

You probably should turn this on. If you're going to run HMMER on a
multiprocessor machine, you want this option. Even if you aren't, it
doesn't hurt. (Some platforms don't support POSIX threads, so
mulltithreading is not configured by default.)

If you turn on threads support, by default, the multithreaded HMMER
programs \prog{hmmcalibrate}, \prog{hmmsearch}, and \prog{hmmpfam}
will use \emph{all} available CPUs. You can alter this behavior in
three ways: by defining \prog{HMMER\_NCPU} in the \prog{src/config.h}
file that \prog{configure} generates (see ``The config.h header
file'', below); by defining \prog{HMMER\_NCPU} as a shell environment
variable where you run HMMER (see ``Shell environment variables
understood by HMMER'', below), or using the \prog{--cpu <x>} option
with the multithreaded programs (see the ``Manual pages'' section
later in the guide).

\subsubsection{turning on PVM support for clusters}

HMMER can also run in parallel on distributed clusters, using the
Parallel Virtual Machine (PVM) message passing library.  To enable
the optional PVM support, use the \prog{--enable-pvm} flag:

\user{./configure --enable-pvm}

You need to have PVM installed first, and the environment variables
\prog{PVM\_ROOT} and \prog{PVM\_ARCH} must be set appropriately, so
HMMER can find the PVM headers and library.  You can obtain the free
PVM library from Oak Ridge National Laboratory, at
\htmladdnormallink{http://www.csm.ornl.gov/pvm/}{http://www.csm.ornl.gov/pvm/}.

\subsubsection{turning on LFS support for files $>$2 GB}

HMMER can access large files ($>$ 2 GB) even on 32-bit operating
systems, using Large File Summit (LFS) extensions.  Most modern UNIX
systems have LFS extensions. To enable this optional code, use the
\prog{--enable-lfs} flag:

\user{./configure --enable-lfs}

\subsubsection{turning on Altivec optimization for Macintosh PowerPC}

HMMER contains code specifically optimized for the Macintosh PowerPC,
contributed by Erik Lindahl at Stanford University. This code is about
four times faster than the normal code. To enable this optional code,
use the \prog{--enable-altivec} flag:

\user{./configure --enable-altivec}

The \prog{configure} script will automatically check whether you're on
a machine that supports Altivec instructions or not, so it doesn't
hurt to try to turn this on if you're not sure.

\subsubsection{other options, used in development code}

The remaining two features selectable with the \prog{configure} script
will only be useful to you if you're a developer.

The \prog{--enable-ccmalloc} option configures HMMER to be
instrumented by the \prog{ccmalloc} memory checking library, a free
software library from Armin Biere at ETH Zurich.  You have to have
\prog{ccmalloc} installed.  \prog{ccmalloc} can be obtained from
\htmladdnormallink{http://www.inf.ethz.ch/personal/biere/projects/ccmalloc/}
{http://www.inf.ethz.ch/personal/biere/projects/ccmalloc/}.

The \prog{--enable-debugging} option is the same as passing
\prog{CFLAGS=-g} in the environment. \prog{--enable-debugging} can
also take a numeric argument from 1-3, which sets the verbosity of
additional debugging output from all the
programs. \prog{--enable-debugging=1} is the least, and
\prog{--enable-debugging=3} is the most verbose. There is no telling
what will spew out of the programs if you set these, because I tend to
change the debugging output on a whim.

\subsubsection{example configuration}

The Intel GNU/Linux version installed in St. Louis is configured as
follows:

\user{env CC=icc CFLAGS="-O -static" ./configure --enable-threads --enable-pvm --enable-lfs --prefix=/usr/seshare/ --exec-prefix=/usr/seshare/`uname`}

\subsection{The config.h header file}

The \prog{configure} script generates a \prog{src/config.h} file that
contains most of the configuration information that the source code
uses. You don't have to edit this file.  There are two things that are
worth knowing about, though.

\subsubsection{controlling memory footprint with \prog{RAMLIMIT}}

HMMER has two different implementations of its alignment algorithms.
One implementation is fast, but consumes memory proportional to the
product of the query and target sequence lengths. The other
implementation is about two-fold slower, but consumes memory
proportional to the \emph{sum} of the query and target lengths, which
can be orders of magnitude less memory than the normal algorithm.
Having both algorithms allows HMMER to efficiently deal with almost
any size of query model and target sequence.

HMMER switches to the small memory variant algorithm when the normal
algorithm would take more than a certain number of megabytes of
memory. This threshold is set by the definition of RAMLIMIT in
\prog{config.h}. By default, RAMLIMIT is 32. 

On a multiprocessor, this means 32 MB per thread, e.g. per CPU. On a
64-processor machine, for example, HMMER expects by default to have 2
GB of memory (64 $\times$ 32 MB).

This number only affects the switchover point from the normal
algorithm to the small memory variants, so it is not a perfect
estimate of the actual total memory consumed by HMMER. But since the
alignment algorithms dominate HMMER's memory usage, the correspondence
is pretty good.

Therefore, if you know you will be running HMMER on a machine with
less than 32 MB of memory per CPU, you might want to reduce RAMLIMIT
before compiling. (Even 0 will work -- it will force the small memory
variant algorithms to be used for all target sequences.) Or, if you
know you will always have a lot of memory available, you can increase
RAMLIMIT and the programs will run faster, depending on how many more
target sequences get processed by the faster algorithms.

\subsubsection{limiting the default number of processors with \prog{HMMER\_NCPU}}

By default, the multithreaded version of HMMER (e.g. what you get with
passing \prog{--enable-threads} to \prog{configure}) will use
\emph{all} available CPUs. This might not be socially acceptable
behavior, especially on a shared resource. 

You can compile in a different default limit by defining the
\prog{HMMER\_NCPU} variable in \prog{config.h}.

If you're compiling HMMER for a large multiprocessor system that's
shared by a number of people, and you want most HMMER jobs to access
only a small number of processors by default, setting
\prog{HMMER\_NCPU} is a good idea.

This ``limit'' is only a default. It can still be overridden in the
shell environment by setting an environment variable
\prog{HMMER\_NCPU}, or by the user on the command line using a
\prog{--cpu <x>} option.

\subsubsection{the other stuff in \prog{config.h}}
\label{section:alphabet-config}

The other stuff in \prog{config.h} is dangerous to change, unless you
know what you're doing, or are willing to learn. One example of when
you might change it is if you're warping HMMER code to work on
non-biological alphabets. HMMER can be configured for arbitrary symbol
alphabets by changing key constants in \prog{config.h}, plus code in
\prog{globals.h} where the alphabet is declared, and \prog{alphabet.c}
where the alphabet gets set up. I haven't done this myself but it is
apparently fairly straightforward. HMMER has reportedly been adapted
to speech recognition, recorded music reconstruction, and automobile
engine telemetry applications, and possibly others that I haven't
heard about.

\subsection{Running make}

When you run \prog{configure}, it also creates the Makefiles that will
actually compile and install the code.  There are a variety of make
targets in the toplevel Makefile. Typing

\user{make}

is equivalent to \prog{make all} - it compiles all the source code,
and leaves the executables in the \prog{src/} subdirectory.  The other
make targets that you might use, in the order you're likely to use
them, are:

\begin{wideitem}
\item[\textbf{check}] Runs the testsuite; verifies that HMMER has
compiled correctly.

\item[\textbf{install}] Installs the programs in BINDIR and
the man pages in MANDIR. Creates the target directories if
they don't already exist.

\item[\textbf{clean}] Cleans up the directory, removing files
generated in the compilation process, while leaving the original
distribution files, plus the Makefiles and the programs.

\item[\textbf{uninstall}] Remove a previously installed HMMER,
reversing a previous \prog{make install}. This assumes you left the
toplevel Makefile untouched since you did the \prog{make install}.

\item[\textbf{distclean}] Like 'make clean', but removes everything
that isn't part of a pristine source distribution. Use this only if
you're going to start configuring and compiling HMMER all over again.
\end{wideitem}

\subsection{Shell environment variables understood by HMMER}
\label{section:environment}

HMMER is built to coexist peacefully with the BLAST suite of database
search programs \cite{Altschul91,Altschul97}. HMMER reads the
following environment variables (the examples given use UNIX csh
syntax):

\begin{wideitem}
\item [\emprog{BLASTDB}] Location of sequence databases that
	\prog{hmmsearch} will look in, in addition to the current
	working directory.
	Multiple directories are allowed, separated by colons. A
	trailing slash on each path is important to BLAST, but not to HMMER.\\
	Examples: \\
	\user{setenv BLASTDB /nfs/databases/}\\
        \user{setenv BLASTDB /nfs/databases/:/nfs/moredatabases/}

\item [\emprog{BLASTMAT}] Location of substitution matrices that
	\prog{hmmbuild --pam} (the PAM prior option) can read.
	Although HMMER can parse a colon-separated list, BLAST must
	have a single directory path here.
	Example:\\
	\user{setenv BLASTMAT /nfs/databases/matrix/}

\item [\emprog{HMMERDB}] Location of HMMs, PFAM, or other HMMER
	specific data files. Any program that reads an HMM file
	looks in both HMMERDB and the current working directory.
	Multiple directories are allowed, colon-separated.
	Examples:\\
	\user{setenv HMMERDB /usr/local/lib/hmmer/}\\
	\user{setenv HMMERDB /usr/local/lib/hmmer/:/nfs/databases/pfam/}

\item [\emprog{HMMER\_NCPU}] On multiprocessors that support POSIX
	threads (this includes almost all modern UNIX multiprocessors;
	for example, SGI Origin servers), the programs
	\prog{hmmcalibrate}, \prog{hmmpfam}, and \prog{hmmsearch}
	run as parallelized, multithreaded applications.
	Normally they will take over all available CPUs in the machine.
	HMMER\_NCPU sets a maximum number of CPUs to utilize,
	so HMMER searches are ``good citizens'', leaving some
 	CPU power for other jobs. An example of configuring
	HMMER to use only 16 processors on a 32-processor Origin:\\
	\user{setenv HMMER\_NCPU 16}
\end{wideitem}

If you have installed BLAST, you probably already have the two BLAST
environment variables set in system-wide or user-specific .cshrc
files.

All four variables are optional. HMMER looks for any sequence file
first in \prog{BLASTDB}, then in the current working directory. It
looks for any HMM file first in \prog{HMMERDB}, then in the current
working directory. Thus if these are set up, you can simplify command
lines to:

\user{hmmpfam Pfam my.query}\\
\user{hmmsearch my.hmm swiss35}

instead of:

\user{hmmpfam   /some/long/path/to/databases/Pfam my.query}\\
\user{hmmsearch my.hmm /some/long/path/to/databases/swiss35}

HMMER only uses BLAST scoring matrices with the \prog{hmmbuild
--pamprior} option. If you use this, the matrix is looked for in the
\prog{BLASTMAT} directory first, then in the current directory.

The \prog{HMMER\_NCPU} environment variable takes precedence over any
\prog{HMMER\_NCPU} processor number limit defined in \prog{config.h}
-- see the discussion of this in the previous section.

\subsection{Configuring a PVM cluster for HMMER}

The following only applies to people installing HMMER on a distributed
cluster using PVM. I assume you're already reasonably familiar with
PVM applications (at least, as familiar as I am -- I'm only familiar
enough with it to make it work), and that you have PVM installed.

Designate one machine as the ``master'', and the other machines as
``slaves''. You will start your HMMER process on the master, and the
master will spawn jobs on the slaves using PVM.\footnote{HMMER will
not parallelize on PVM capable systems that cannot run a strict
master-slave model, that is, where the master is running a different
program than its slaves. For example, Cray T3E systems are in
principle PVM-capable, but only when all processors instantiate the
same program, and that program figures out for itself whether it's the
master calling the shots, or a slave communicating back to the
master. This style of message passing parallelization is more MPI
like, and becoming more standard -- but is not supported by HMMER.}

Install PVM on the master and all the slaves. On the master, make sure
the environment variables \prog{PVM\_ROOT} and \prog{PVM\_ARCH} are
set properly (ideally, in a system-wide .cshrc file). You may also
want to have \prog{PVM\_RSH} set to \prog{ssh} (we do).

If you're using rsh to connect to the slaces, Add the master's name to
your .rhosts or /etc/hosts.equiv file on the slaves, so the slaves
accept passwordless rsh connections from the master. Or, if you're
using ssh, do the equivalent with ssh authentication (and make sure
you have \prog{PVM\_RSH} set to \prog{ssh} in the environment). Test
this by rsh'ing or ssh'ing into the slaves from the master's command
line, independent of PVM and HMMER.

Put copies of HMMER executables in a directory on the master and all
the slaves. For each PVM-capable program (\prog{hmmcalibrate},
\prog{hmmpfam}, and \prog{hmmsearch}, there is a corresponding slave
PVM program (\prog{hmmcalibrate-pvm}, \prog{hmmpfam-pvm}, and
\prog{hmmsearch-pvm}). The master machine needs copies of all the HMMER
programs, including the slave PVM programs.  The slaves only need
copies of the three slave PVM programs. (You never need to start the
slave programs yourself; PVM does that. You just need to make sure
they're installed where PVM can see them.)

The PVM implementation of \prog{hmmpfam} needs a copy of any HMM
databases you may search to be installed on the master and every
slave. All HMM databases must be indexed with \prog{hmmindex}. The
reason is that \prog{hmmpfam} is I/O bound; the PVM implementation
can't distribute an HMM database fast enough over a typical cluster's
Ethernet. Instead, each PVM node accesses its own copy of the HMM
database, distributing the I/O load across the nodes.
\prog{hmmcalibrate} and \prog{hmmsearch}, in contrast, are
freestanding. Only the master node needs to be able to access any HMM
and/or sequence files.

Write a PVM hostfile for the cluster. Specify the location of the
HMMER executables using the ep= directive. Specify the location of
\prog{pvmd} on the slaves using the dx= directive (alternatively, you
can make sure \prog{PVM\_ROOT} and \prog{PVM\_ARCH} get set properly
on the slaves). For the slaves, use the wd= directive to specify the
location of the HMM databases for \prog{hmmpfam} (alternatively, you
can make sure HMMERDB gets set properly on the slaves). Use the sp=
directive to tell HMMER how many processors each node has (and hence,
how many independent PVM processes it should start); sp=1000 means 1
CPU, sp=2000 means 2 CPUs, sp=4000 means 4 CPUS, etc.

Start the PVM by typing

   \user{pvm hostfile} 

(where ``hostfile'' is the name of your hostfile) on the master. Make
sure all the nodes started properly by typing 

   \user{conf}

at the PVM
console prompt.  Type 

   \user{quit} 

to exit from the PVM console, which leaves the PVM running in the
background.  You should only need to start PVM once. (We have a PVM
running continuously on our network right now, waiting for HMMER
jobs.)

Once PVM is running, at any time you can run HMMER programs on the
master and exploit your PVM, just by adding the option \prog{--pvm};
for instance,

\user{hmmpfam --pvm Pfam my.query} 

parallelizes a search of a query sequence in the file my.query against
the Pfam database.

Once PVM is properly configured and your slave nodes have the required
slave programs (and databases, in the case of \prog{hmmpfam}), the
only difference you will notice between the serial and the PVM version
is a (potentially massive) increase in search speed. Aside from the
addition of the \prog{--pvm} option on the command line, all other
options and input/output formats remain identical.

\subsubsection{example of a PVM cluster}

The St. Louis Pfam server runs its searches using HMMER on a PVM
cluster called Wulfpack. I'll use it as a specific example of
configuring a PVM cluster. It's a little more intricate than you'd
usually need for personal use, just because of the details of running
PVM jobs in a standalone way from CGI scripts on a Web server.
\footnote{These instructions were made for an older version of
Wulfpack; machine names and numbers have changed since then.}

The master node is the Web server itself, \prog{fisher}. The slave
nodes are eight dual processor Intel/Linux boxes called \prog{wulf01}
through \prog{wulf08}. 

PVM 3.3.11 is installed in \prog{/usr/local/pvm3} on the master and
all the slaves.

On fisher, all HMMER executables are installed in
\prog{/usr/local/bin}.  On the wulf slave nodes, the three PVM slave
executables are installed in \prog{/usr/local/wulfpack}.

Pfam and PfamFrag, two Pfam databases, are installed on the wulf slave
nodes in \prog{/usr/local/wulfpack.} They are converted to binary
format using \prog{hmmconvert -b}, then indexed using \prog{hmmindex}.
(Using binary format databases is a performance win for \prog{hmmpfam}
searches, because \prog{hmmpfam} is I/O bound and binary HMM databases
are smaller.)

An \prog{ls} of \prog{/usr/local/wulfpack} on any wulf node looks like:

\begin{sreoutput}
[eddy@wulf01 /home]$ ls /usr/local/wulfpack/
Pfam             PfamFrag         hmmcalibrate-pvm   hmmsearch-pvm
Pfam.ssi         PfamFrag.ssi     hmmpfam-pvm
\end{sreoutput}

The PVM hostfile for the cluster looks like:
\begin{sreoutput}
# Config file for Pfam Web server PVM
#
* ep=/usr/local/bin sp=1000
fisher.wustl.edu
* lo=pfam dx=/usr/local/pvm3/lib/pvmd ep=/usr/local/wulfpack sp=2000
wulf01
wulf02
wulf03
wulf04
wulf05
wulf06
wulf07
wulf08
\end{sreoutput}

A wrinkle specific to configuring Web servers: the web server is
running HMMER as user ``nobody'' because it's calling HMMER from a CGI
script. We can't configure a shell for ``nobody'' on the slaves, so we
create a dummy user called ``pfam'' on each wulf node.  The lo=
directive in the PVM hostfile is telling the master to connect to the
slaves as user ``pfam''. On each slave, there is a user ``pfam'' with
a .rhosts that looks like:
\begin{sreoutput}
   fisher nobody
   fisher.wustl.edu nobody
\end{sreoutput}
which tells the wulf node to allow fisher's user ``nobody'' 
to connect to the wulf node as user ``pfam''.

Also note how we use the sp= directive to tell HMMER (via PVM) that
the wulf nodes are dual processors. fisher is actually a dual
processor too, but by setting sp=1000, HMMER will only start one PVM
process on it (leaving the other CPU free to do all the things that
keep Web servers happy).

The trickiest thing is making sure \prog{PVM\_ROOT} and
\prog{PVM\_ARCH} get set properly.  For my own private PVM use, my
\prog{.cshrc} contains the lines:
\begin{sreoutput}
	setenv PVM_ROOT    /usr/local/pvm3
	setenv PVM_ARCH    `$PVM_ROOT/lib/pvmgetarch`
\end{sreoutput}

But for the web server PVM, it's a little trickier. At boot time, we
start the Web server's pvmd as user ``nobody'' on fisher using a local
init script, \prog{/etc/rc.d/init.d/pvm\_init}. With its error
checking deleted for clarity, this script basically looks like:

\begin{sreoutput}
#!/bin/sh
wulfpack_conf=/home/www/pfam/pfam-3.1/wulfpack.conf
. /usr/local/pvm3/.pvmprofile
$PVM_ROOT/lib/pvmd $wulfpack_conf >/dev/null &
\end{sreoutput}

We call this at boot time by adding the line \prog{su nobody -c "sh
/etc/rc.d/init.d/pvm\_init"} to our \prog{rc.local}
file. \prog{.pvmprofile} is a little PVM-supplied script that properly
sets \prog{PVM\_ROOT} and \prog{PVM\_ARCH}, and \prog{wulfpack.conf}
is our PVM hostfile.

The relevant lines of the CGI Perl script that runs HMMER jobs from
the Web server (again, heavily edited for clarity) are:

\begin{sreoutput}
# Configure environment for PVM
$ENV{'HMMERDB'}    = "/usr/local/wulfpack:/home/www/pfam/data/"
$ENV{'PVM_EXPORT'} = "HMMERDB";
$output = `/usr/local/bin/hmmpfam --pvm Pfam /tmp/query`;
\end{sreoutput}

The trick here is that we export the HMMERDB environment variable via
PVM, so the PVM processes on wulf nodes will know where to find their
copy of Pfam.

\begin{srefaq}{Why is HMMER crashing when I try to use it under PVM?}
It appears that the PVM port may not be as good as it should be. It
happens to work well in St. Louis. It is used 24/7 on the servers
backing the St. Louis Pfam search server at
\htmladdnormallink{http://pfam.wustl.edu}{http://pfam.wustl.edu/}.
However, I receive occasional reports of problems at other
sites. John Blanchard at Incyte has contributed a number of
improvements to the PVM code, but despite his generous hard work, scattered problems
persist. Unfortunately, PVM is painful for me to debug, and I have been
unable to reproduce any of these latest problems in St. Louis. Since it works
fine for me, I can't reproduce the issues; and nobody has sent me a small,
detailed, and reproducible bug report, so I'm not able to track these
phantoms down. But it seems only fair to warn you that this is an area
of the code where bugs may lurk.
\end{srefaq}

\begin{srefaq}{Does HMMER support MPI?}
Not yet.
\end{srefaq}


\subsection{Contributing binary distros back to the HMMER web site}

Ah, what a helpful net citizen you are! If you're compiling from
source on a platform for which HMMER binaries are not yet available,
here's how to contribute your binary distro back to us.

First, choose an informative name for your platform, like
\prog{intel-linux}, \prog{sgi-irix64}, or
\prog{powerpc-apple-osx}. The name should reflect what platforms your
binaries will run on, of course.  Your binary distro will have this
string as part of its name.

A safe bet is to use the output of GNU's \prog{config.guess} program, in
HMMER's top level directory. On the platform I build \prog{intel-linux}
binaries on, this gives:

\user{sh config.guess}\\
\response{i686-pc-linux-gnu}

The \prog{config.guess} output is often overly specific. I generally
shorten that in some reasonable way to arrive at my names.

In the examples below, I'll use \prog{<platform>} to mean the string
you choose. I'll use 2.3.2 for the HMMER version number; you need to
substitute with the correct version number, if you're not building
2.3.2.

Now you can either build the binary distro manually, or by using the
\prog{nodebuild} script, as follows:

\subsubsection{Manual, step-by-step instructions}

\begin{itemize}
\item Get the HMMER source code, \prog{hmmer-2.3.2.tar.gz}.

\item Unpack and rename the directory to \prog{hmmer-2.3.2.bin.<platform>}.

\user{tar zxf hmmer-2.3.2.tar.gz}\\
\user{mv hmmer-2.3.2 hmmer-2.3.2.bin.<platform>}

\item Go into the \prog{hmmer-2.3.2} directory; follow the compilation
   instructions. For instance, a default build would be:

\user{./configure}\\
\user{make}\\
\user{make check}

(You don't need to \prog{make install}, though it won't matter if you
do.)

The \prog{make check} should report no problems. If it does, you can
report it as a bug.

\item Package the binary distribution.

\user{make bindist}\\
\user{cd ..}\\
\user{tar cf hmmer-2.3.2.bin.<platform>.tar hmmer-2.3.2.bin.<platform>}\\
\user{gzip hmmer-2.3.2.bin.<platform>.tar}

\item Cleanup. 

\user{rm -rf hmmer-2.3.2.bin.<platform>}
\end{itemize}


Skip to ``Sending the files'', below.


\subsubsection{Automatic, using the \prog{nodebuild} script}

The \prog{nodebuild} script is part of the automation on our compile
farm.  You can download a copy from
\url{ftp://ftp.genetics.wustl.edu/pub/eddy/software/nodebuild}.

The top of the \prog{nodebuild} file contains some documentation for
using \prog{nodebuild} to build binary distros on a remote host
(e.g. in our compile farm, we drive mass compilations from a single
server).  To build on a local host is even simpler, where the command
line is:

\begin{sreoutput}
   ./nodebuild <dir> <pkg> <platform> <configopts> <makeopts>
   
   <dir>         = where the HMMER tarball is
   <pkg>         = e.g., hmmer-2.3.2
   <platform>    = e.g., intel-linux
   <configopts>  = options to ./configure; e.g. \"--enable-threads\"
   <makeopts>    = options to make: e.g. CC=gcc
\end{sreoutput}

There's some silly business with double quotes here, to get arguments
into the script properly without having your shell screw them up. Note
that the configopts have to be wrapped in backslash-protected double
quotes if there's more than one option. To include *no* configure
options, use \verb+""+ to leave a blank field there.  The makeopts
don't need to be protected with double quotes, but any multiword
options in makeopts do.

Example:

\begin{sreoutput}
     > ./nodebuild /tmp hmmer-2.3.2 intel-linux \"--enable-threads\
          --enable-lfs\" CC=gcc CFLAGS=\"-O2 -Wall\" 
\end{sreoutput}

The \prog{nodebuild} script creates a log, \prog{hmmer-2.3.2.<platform>.log}.

So, let's say you're in your home directory, \prog{/home/eddy}. You
would do something like the following:
   
\begin{itemize}
\item Get the source, \prog{hmmer-2.3.2.tar.gz}, and the \prog{nodebuild} script.

\user{ls}\\
\response{hmmer-2.3.2.tar.gz  nodebuild}\\
\user{pwd}\\
\response{/home/eddy}

\item Run nodebuild.

\begin{sreoutput}
     > ./nodebuild /home/eddy hmmer-2.3.2 intel-linux\
       \"--enable-threads --enable-lfs\" CC=gcc CFLAGS=\"-O2 -Wall\"
\end{sreoutput}

\item Voila.

\user{ls}\\
\response{hmmer-2.3.2.tar.gz}\\
\response{hmmer-2.3.2.<platform>.log}\\
\response{hmmer-2.3.2.bin.<platform>.tar.gz}\\
\response{nodebuild}
\end{itemize}

\subsubsection{Sending the files}

Send me the \prog{.tar.gz} file (and the \prog{.log} file, if you used
\prog{nodebuild}), either by putting it on an FTP site or by sending as
an email attachment.

If you had to make any changes in the source code, please let me
know. If you send me a patch file too, that's great; but I can also
extract my own patches from your binary distro.









