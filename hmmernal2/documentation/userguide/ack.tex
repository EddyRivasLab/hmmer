\section{Acknowledgements and history}

HMMER 1 was developed on slow weekends in the lab at the MRC
Laboratory of Molecular Biology, Cambridge UK, while I was a postdoc
with Richard Durbin and John Sulston. I thank the Human Frontier
Science Program and the National Institutes of Health for their
remarkably enlightened support at a time when I was really supposed to
be working on the genetics of neural development in \emph{C. elegans}.

HMMER 1.8, the first public release of HMMER, came in April 1995,
shortly after I moved to Washington University in St. Louis. A few
bugfix releases followed. A number of more serious modifications and
improvements went into HMMER 1.9 code, but 1.9 was never
released. Some versions of HMMER 1.9 did inadvertently escape
St. Louis and make it to some genome centers, but 1.9 was never
documented or supported. HMMER 1.9 collapsed under its own weight in
1996.

HMMER 2 is a nearly complete rewrite, based on the new Plan 7 model
architecture. Implementation was begun in November 1996. I thank the
Washington University Dept. of Genetics, the NIH National Human Genome
Research Institute, and Monsanto for their support during this time.
Also, I thank the Biochemistry Academic Contacts Committee at Eli
Lilly \& Co. for a gift that paid for the trusty Linux laptop on which
much of HMMER 2 was written. The laptop was indispensable. Far too
much of HMMER was written in coffee shops, airport lounges,
transoceanic flights, and Graeme Mitchison's kitchen. The source code
still contains a disjointed record of where and when various bits were
written.

HMMER has now settled into a comfortable middle age, like its author;
still actively maintained, though dramatic changes are increasingly
unlikely. HMMER 2.1.1 was the stable release for three years, from
1998-2001.  HMMER 2.2g was intended to be a beta release, but became
the \emph{de facto} stable release for two more years, 2001-2003. The
latest release, 2.3, has been assembled in spring 2003.

The MRC-LMB computational molecular biology discussion group
contributed many ideas to HMMER. In particular, I thank Richard
Durbin, Graeme Mitchison, Erik Sonnhammer, Alex Bateman, Ewan Birney,
Gos Micklem, Tim Hubbard, Roger Sewall, David MacKay, and Cyrus
Chothia. 

The UC Santa Cruz HMM group, led by David Haussler and including
Richard Hughey, Kevin Karplus, Anders Krogh (now back in Copenhagen)
and Kimmen Sj\"{o}lander, has been a source of knowledge, friendly
competition, and occasional collaboration. All scientific competitors
should be so gracious. The Santa Cruz folks have never complained (at
least in my earshot) that HMMER started as simply a re-implementation
of their original ideas, just to teach myself what HMMs were.

Sequence format parsing ({\tt sqio.c}) in HMMER is derived from an
early release of the {\tt READSEQ} package by Don Gilbert, Indiana
University. Thanks to Don for an excellent piece of software; and
apologies for the years of mangling I've put it through since I
obtained it in 1992. The file {\tt hsregex.c} is a derivative of Henry
Spencer's regular expression library; thanks, Henry. Several
miscellaneous functions in {\tt sre\_math.c} are taken from public
domain sources and are credited in the code's comments. {\tt masks.c}
includes a modified copy of the XNU source code from David States and
Jean-Michel Claverie.

John Blanchard (Incyte Pharmaceuticals) made several contributions to
the PVM port; the remaining bugs are my own dumb fault.  Dave Cortesi
(Silicon Graphics) contributed much useful advice on the POSIX threads
implementation. The blazing fast Altivec port to Macintosh PowerPC was
contributed in toto by Erik Lindahl at Stanford.

In many other places, I've reimplemented algorithms described in the
literature. These are too numerous to credit and thank here. The
original references are given in the comments of the code. However,
I've borrowed more than once from the following folks that I'd like to
be sure to thank: Steve Altschul, Pierre Baldi, Phillip Bucher, Warren
Gish, Steve and Jorja Henikoff, Anders Krogh, and Bill Pearson.

HMMER is primarily developed on GNU/Linux machines, but is tested on a
variety of hardware. Compaq, IBM, Intel, Sun Microsystems, Silicon
Graphics, Hewlett-Packard, and Paracel have provided the generous
hardware support that makes this possible. I owe a large debt to the
free software community for the development tools I use: an incomplete
list includes GNU gcc, gdb, emacs, and autoconf; Cygnus' and others'
egcs compiler; Conor Cahill's dbmalloc library; Bruce Perens'
ElectricFence; Armin Biere's ccmalloc; the cast of thousands that
develops CVS, the Concurrent Versioning System; Larry Wall's perl;
LaTeX and TeX from Leslie Lamport and Don Knuth; Nikos Drakos'
latex2html; Thomas Phelps' PolyglotMan; Linus Torvalds' Linux
operating system; and the folks at Red Hat Linux and Mandrake Linux.

Finally, I will cryptically thank Dave ``Mr. Frog'' Pare and Tom
``Chainsaw'' Ruschak for a totally unrelated free software product
that was historically instrumental in HMMER's development -- for
reasons that are best not discussed while sober.

\label{manualend}