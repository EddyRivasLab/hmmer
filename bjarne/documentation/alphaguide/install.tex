\section{Installation}
\label{section:installation}

\subsection{Quick installation instructions}

Download \prog{hmmer-3.0b2.tar.gz} from
\url{http://hmmer.org/}, or directly from
\url{ftp://selab.janelia.org/pub/software/hmmer3/hmmer-3.0b2.tar.gz};
untar; and change into the newly created directory \prog{hmmer-3.0b2}:

\user{wget ftp://selab.janelia.org/pub/software/hmmer3/hmmer-3.0b2.tar.gz}\\
\user{tar xf hmmer-3.0b2.tar.gz}\\
\user{cd hmmer-3.0b2}

The alpha test code includes precompiled binaries for x86/Linux
platforms. These are in the \prog{binaries} directory. You can just
stop here if you like, if you're on a x86/Linux machine and you want
to try the programs out without installing them.

To compile new binaries from source, do a standard GNUish build:

\user{./configure}\\ 
\user{make}

To compile and run a test suite to make sure all is well, you can
optionally do:

\user{make check}

All these tests should pass.

You don't have to install HMMER programs to run them. The newly
compiled binaries are now in the \prog{src} directory; you can run
them from there. To install the programs and man pages somewhere on
your system, do:

\user{make install} 

By default, programs are installed in \prog{/usr/local/bin} and man
pages in \prog{/usr/local/man/man1/}. You can change \prog{/usr/local}
to any directory you want using the \prog{./configure --prefix}
option, as in \prog{./configure --prefix /the/directory/you/want}.

If you have the Intel C compiler \prog{icc}, we strongly recommend
that you use it (instead of \prog{gcc}, for example), for performance
reasons, by specifying {CC=icc} either in your environment or on the
\user{./configure} command line. 

For example, on our systems, we would do:

\user{./configure CC=icc LDFLAGS=-static --prefix=/usr/local/hmmer-3.0/}\\
\user{make}\\
\user{make check}\\
\user{make install}


\subsection{System requirements}

\textbf{Operating system:} HMMER is designed to run on
POSIX-compatible platforms, including UNIX, Linux and MacOS/X.  The
POSIX standard essentially includes all operating systems except
Microsoft Windows.
\footnote{There are add-on products available for making Windows more
  POSIX-compliant and more compatible with GNU-ish configures and
  builds. One such product is Cygwin, \url{http:www.cygwin.com}, which
  is freely available. Although we do not test on Windows platforms,
  we understand HMMER builds fine in a Cygwin environment on Windows.}

The alpha test code includes precompiled binaries for Linux. These
were compiled with the Intel C compiler (\prog{icc}) on an x86\_64
Intel platform running Red Hat Enterprise Linux AS4. We believe they
should be widely portable to different Linux systems. 

We have tested most extensively on Linux, and to a lesser extent on
MacOS/X. We aim to be portable to all other POSIX platforms. We
currently do not develop or test on Windows.


\textbf{Processor:} HMMER3 currently runs effectively only on x86
(IA32/IA64) processors that support the SSE2 vector instruction set.
This includes all Intel processors from the Pentium 4 on, and all AMD
processors from the K8 (Athlon) on. We believe this includes almost
all Intel and AMD x86 processors produced since about 1999.

We aim to be portable to all modern processors. The acceleration
algorithms are designed to be portable despite their use of
specialized SIMD vector instructions. In the future we plan to to
support processors that use VMX/Altivec vector instructions, including
PowerPC processors and the Cell processor, and hope to be able to add
support for the Sun SPARC VIS instruction set as well. We believe that
the code will be able to take advantage of GP-GPUs and FPGAs in the
future.

\textbf{Compiler:} The source code is C, conforming to POSIX and ANSI
C99 standards. It should compile with any ANSI C99 compliant compiler,
including the GNU C compiler \prog{gcc}. We have tested the code
using both the \prog{gcc} and \prog{icc} compilers.

If you compile HMMER from source, we strongly recommend using the
Intel C compiler \prog{icc} rather than \prog{gcc}. \prog{icc} is free
for noncommercial use and heavily discounted for academic use.  HMMER3
makes extensive use of SIMD vector instructions (SSE, Intel's
Streaming SIMD Extensions). \prog{gcc} is not as effective at
compiling SSE code. It produces HMMER programs that are significantly
slower than what \prog{icc} produces.

\textbf{Libraries and other installation requirements:} HMMER includes
a software library called Easel, which it will automatically compile
during its installation process.  By default, HMMER3 does not require
any additional libraries to be installed by you, other than standard
ANSI C99 libraries that should already be present on a system that can
compile C code.

\begin{sidebar}
One of the objectives of the alpha test is to identify portability
issues. If HMMER3 fails to compile and/or run under a POSIX-compliant
OS, on an x86 processor, using an ANSI C99-compliant compiler, please
report the problem.
\end{sidebar}


\subsection{Experimental: MPI support}

Optionally, a couple of programs in HMMER3 include support for MPI
(Message Passing Interface) parallelization. To use MPI, you first
need to have MPI installed, such as OpenMPI (\url{www.open-mpi.org}),
and you add \ccode{--enable-mpi} to the \ccode{./configure} command
line. HMMER3's MPI support is not completely written at this time, and
has not been extensively tested yet. 

One place it's useful is in parallelizing \prog{hmmbuild}.

It's highly advantageous to get \prog{mpicc} to use the Intel C
compiler rather than its default, which is often \prog{gcc}. Different
MPI distributions may have different ways of selecting the C compiler
and its options. OpenMPI can be controlled by environment variables.
For example, in our environment, we currently build HMMER3 for MPI
using:

\user{setenv OMPI\_CC      "icc"}\\
\user{setenv OMPI\_CFLAGS  "-O3"}\\
\user{./configure --enable-mpi --prefix=/usr/local/hmmer3}\\
\user{make}


\subsection{Makefile targets}

\begin{sreitems}{\emprog{distclean}}
\item[\emprog{all}]
  Builds everything. Same as just saying \ccode{make}.

\item[\emprog{check}]
  Runs automated test suites in both HMMER and the Easel library.

\item[\emprog{clean}]
  Removes all files generated by compilation (by
  \ccode{make}). Configuration (files generated by
  \ccode{./configure}) is preserved.

\item[\emprog{distclean}]
Removes all files generated by configuration (by \ccode{./configure})
and by compilation (by \ccode{make}). 

Note that if you want to make a new configuration (for example, to try
an MPI version by \ccode{./configure --enable-mpi; make}) you should
do a \ccode{make distclean} (rather than a \ccode{make clean}), to be
sure old configuration files aren't used accidentally.
\end{sreitems}




