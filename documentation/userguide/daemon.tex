\documentclass[notoc]{tufte-book}    % `notoc` suppresses TL custom TOC, reverts to standard LaTeX

\hyphenation{HMMER}

\title{User's Guide for the HMMER Daemon}

\subtitle{High-performance biological sequence analysis using profile hidden Markov models}

\author{Sean R. Eddy, Nicholas P. Carter}
\subauthor{and the HMMER development team}

\pkgurl{http://hmmer.org}
\pkgversion{3.3dev}   % ./configure replaces HMMER_VERSION
\pkgdate{November 2018}         %    ... and HMMER_DATE

                    % definitions for \maketitle 
\bibliographystyle{unsrtnat-brief}   % customized natbib unsrtnat. Abbrev 3+ authors to ``et al.'' 

\begin{document}
\setcounter{tocdepth}{2}             % 0=chapters 1=sections 2=subsections 3=subsubsections? 4=paragraphs
\newcommand{\UNIrelease}{2018\_02}
\newcommand{\UNInseq}{556,825}

\newcommand{\HMMERversion}{3.2}
\newcommand{\HMMERdate}{June 2018}

\newcommand{\BGLnseq}{4}
\newcommand{\BGLalen}{171}
\newcommand{\BGLmlen}{149}
\newcommand{\BGLgaps}{22}
\newcommand{\BGLeffn}{0.96}
\newcommand{\BGLre}{0.589}

\newcommand{\HMMERfmtversion}{f}
\newcommand{\HMMERsavestamp}{[3.2 | June 2018]}

\newcommand{\SGUevalue}{6.7e-65}
\newcommand{\SGUbitscore}{222.7}
\newcommand{\SGUbias}{3.2}
\newcommand{\SGUorigscore}{225.9}
\newcommand{\SGUdombitscore}{222.6}
\newcommand{\SGUseqname}{MYG\_PHYCD}
\newcommand{\SGUmsvpass}{3.8}
\newcommand{\SGUbiaspass}{17546}
\newcommand{\SGUvitpass}{2368}
\newcommand{\SGUfwdpass}{1127}
\newcommand{\SGUelapsed}{1.1}

\newcommand{\SFSevalue}{6.2e-56}
\newcommand{\SFSbitscore}{173.1}
\newcommand{\SFSdomevalue}{1.1e-16}
\newcommand{\SFSdombitscore}{47.3}
\newcommand{\SFSexpdom}{9.5}
\newcommand{\SFSndom}{9}

\newcommand{\SFSmaxdom}{7}
\newcommand{\SFSmaxdomu}{5}
\newcommand{\SFSmaxsc}{47.3}
\newcommand{\SFSievalue}{1.1e-16}
\newcommand{\SFSuievalue}{6.0e-11}
\newcommand{\SFSdomZ}{784}
\newcommand{\SFSucevalue}{8.4e-14}
\newcommand{\SFSaidx}{1}
\newcommand{\SFSascore}{-1.5}
\newcommand{\SFSaevalue}{0.18}
\newcommand{\SFSauevalue}{141}
\newcommand{\SFSacoords}{395-410}
\newcommand{\SFSbidx}{6}
\newcommand{\SFSbscore}{0.6}
\newcommand{\SFSbevalue}{0.04}
\newcommand{\SFSbuevalue}{31.4}
\newcommand{\SFSbcoords}{1720-1769}
\newcommand{\SFSainsig}{5.0}
\newcommand{\SFSbinsig}{10.1}

\newcommand{\JHUninc}{954}
\newcommand{\JHUnsig}{954}

\newcommand{\NMHafrom}{302390}
\newcommand{\NMHato}{302466}
\newcommand{\NMHbfrom}{302466}
\newcommand{\NMHbto}{302389}
\newcommand{\NMHnres}{660000}
\newcommand{\NMHntop}{330000}
\newcommand{\NMHnssv}{69202}
\newcommand{\NMHfracssv}{10.5}
\newcommand{\NMHnbias}{57239}
\newcommand{\NMHfracbias}{8.7}
\newcommand{\NMHnvit}{5181}
\newcommand{\NMHfracvit}{0.8}
\newcommand{\NMHnfwd}{3187}
    % snippets captured from output, by gen-inclusions.py 

\maketitle

\vspace*{\fill}
\begin{flushleft}
Copyright (C) 1992-2001, Washington University in St. Louis.\vspace{5mm}

Permission is granted to make and distribute verbatim copies of this
manual provided the copyright notice and this permission notice are
retained on all copies.\vspace{5mm}

The HMMER software package is a copyrighted work that may be freely
distributed and modified under the terms of the GNU General Public
License as published by the Free Software Foundation; either version 2
of the License, or (at your option) any later version. Some versions
of HMMER may have been obtained under specialized commercial licenses
from Washington University; for details, see the files COPYING and
LICENSE that came with your copy of the HMMER software.\vspace{5mm}

This program is distributed in the hope that it will be useful, but
WITHOUT ANY WARRANTY; without even the implied warranty of
MERCHANTABILITY or FITNESS FOR A PARTICULAR PURPOSE.\vspace{5mm}

See the Appendix for a copy of the full text of the GNU General Public
License.\vspace{5mm}

\end{flushleft}


\begin{adjustwidth}{}{-1in}          % TL \textwidth is quite narrow. Expand it manually for TOC and man pages.
\tableofcontents                     
\end{adjustwidth}

In addition to command-line homology search tools such as hmmsearch, hmmscan, and phmmer, the HMMER package provides {\em hmmpgmd}, a tool that allows a set of computers to provide a high-performance homology search service that client machines can send searches to and receive results from using internet sockets.  Unlike HMMER's command-line search tools, which read their sequence or HMM databases from disk on every search, hmmpgmd reads its input database(s) once when it starts and caches them in RAM for the duration of its execution.  This avoids the performance bottleneck created by disk I/O bandwidth and the time required to parse input data files, allowing hmmpgmd to take full advantage of many-core CPUs and multi-CPU servers.  Distributing the work of each search across multiple computers further improves performance, allowing many searches to be completed in only a few seconds.

This manual describes the design implementation, and usage of hmmpgmd.  It assumes that the reader is familiar with HMMER and homology search in general; readers unfamiliar with those topics should read the {\em HMMER User's Guide} first.  This manual is intended for individuals who are interested in either running an hmmpgmd server of their own or in writing clients that communicate with an existing hmmpgmd server, and thus assumes a fair amount of familiarity with computer systems and programming.  Biologists who wish to use an hmmpgmd server in their research without the complexity of configuring one themselves should consider using the European Bioinformatics Institute's HMMER server {\tt www.ebi.ac.uk/Tools/hmmer/}, which provides a web interface to hmmpgmd servers that load a number of common genetic databases.
 

\chapter{Overview}


\chapter{Usage}

\chapter{Daemon-Client Interface}
Client machines use internet sockets to send commands to and receive results from a daemon's master node.  When a client opens a connection to the master node's client communication port (port 51371 by default), the master node forks a thread to manage the connection with the client.  This thread configures a socket to communicate with the client and then repeatedly calls the {\tt clientside\_loop} function, which monitors the socket for commands from the client, until either the client detatches from the port or the daemon shuts down.  This approach allows multiple clients to connect to a daemon simultaneously without interfering with each other, although requests from one client may impact the amount of time it takes for the daemon to respond to requests from other clients.

\section{Daemon Command Format}
Commands from a client to a daemon are variable-length sequences of ASCII text that are terminated by a line containing only two forward slashes ("{\tt //}").  When a command arrives from a client, the daemon reads bytes from the appropriate socket into a buffer until it sees the end-of-command sequence, growing the buffer as necessary\sidenote{This is a security vulnerability that should be addressed in HMMER4, as it allows an adversarial or erroneous client to consume arbitrary amounts of RAM, potentially exceeding the capacity of the master node.}, and then parses the contents of the buffer in order to execute the command.   

\section{Search Results Format}
The results from each search are split across two sockets messages.  The first is a fixed-length {\tt HMMD\_SEARCH\_STATUS} structure that contains two fields: a {\em status} field that contains any error information from the search, and a {\em msg\_size} field, which tells the client how large (in bytes) the second message will be.  The format of the second message depends on whether any errors were encountered during the search.  If an error occurs, the second message is simply a text string containing a description of the error.  In this case, the {\em status} field of the first message contans one of the Easel error codes, and the {\tt msg\_length} field contains the length of the error description string, including its termination character.

If the search succeeds, the {\tt status} field of the first message is set to "eslOK\sidenote{This assignment is done by the {\tt init\_results} function.}" and the second message contains the results of the search.  This message begins with   


\section{Acknowledgements and history}

HMMER 1 was developed on slow weekends in the lab at the MRC
Laboratory of Molecular Biology, Cambridge UK, while I was a postdoc
with Richard Durbin and John Sulston. I thank the Human Frontier
Science Program and the National Institutes of Health for their
remarkably enlightened support at a time when I was really supposed to
be working on the genetics of neural development in \emph{C. elegans}.

HMMER 1.8, the first public release of HMMER, came in April 1995,
shortly after I moved to Washington University in St. Louis. A few
bugfix releases followed. A number of more serious modifications and
improvements went into HMMER 1.9 code, but 1.9 was never
released. Some versions of HMMER 1.9 did inadvertently escape
St. Louis and make it to some genome centers, but 1.9 was never
documented or supported. HMMER 1.9 collapsed under its own weight in
1996.

HMMER 2 is a nearly complete rewrite, based on the new Plan 7 model
architecture. Implementation was begun in November 1996. I thank the
Washington University Dept. of Genetics, the NIH National Human Genome
Research Institute, and Monsanto for their support during this time.
Also, I thank the Biochemistry Academic Contacts Committee at Eli
Lilly \& Co. for a gift that paid for the trusty Linux laptop on which
much of HMMER 2 was written. The laptop was indispensable. Far too
much of HMMER was written in coffee shops, airport lounges,
transoceanic flights, and Graeme Mitchison's kitchen. The source code
still contains a disjointed record of where and when various bits were
written.

HMMER has now settled into a comfortable middle age, like its author;
still actively maintained, though dramatic changes are increasingly
unlikely. HMMER 2.1.1 was the stable release for three years, from
1998-2001.  HMMER 2.2g was intended to be a beta release, but became
the \emph{de facto} stable release for two more years, 2001-2003. The
latest release, 2.3, has been assembled in spring 2003.

The MRC-LMB computational molecular biology discussion group
contributed many ideas to HMMER. In particular, I thank Richard
Durbin, Graeme Mitchison, Erik Sonnhammer, Alex Bateman, Ewan Birney,
Gos Micklem, Tim Hubbard, Roger Sewall, David MacKay, and Cyrus
Chothia. 

The UC Santa Cruz HMM group, led by David Haussler and including
Richard Hughey, Kevin Karplus, Anders Krogh (now back in Copenhagen)
and Kimmen Sj\"{o}lander, has been a source of knowledge, friendly
competition, and occasional collaboration. All scientific competitors
should be so gracious. The Santa Cruz folks have never complained (at
least in my earshot) that HMMER started as simply a re-implementation
of their original ideas, just to teach myself what HMMs were.

Sequence format parsing ({\tt sqio.c}) in HMMER is derived from an
early release of the {\tt READSEQ} package by Don Gilbert, Indiana
University. Thanks to Don for an excellent piece of software; and
apologies for the years of mangling I've put it through since I
obtained it in 1992. The file {\tt hsregex.c} is a derivative of Henry
Spencer's regular expression library; thanks, Henry. Several
miscellaneous functions in {\tt sre\_math.c} are taken from public
domain sources and are credited in the code's comments. {\tt masks.c}
includes a modified copy of the XNU source code from David States and
Jean-Michel Claverie.

John Blanchard (Incyte Pharmaceuticals) made several contributions to
the PVM port; the remaining bugs are my own dumb fault.  Dave Cortesi
(Silicon Graphics) contributed much useful advice on the POSIX threads
implementation. The blazing fast Altivec port to Macintosh PowerPC was
contributed in toto by Erik Lindahl at Stanford.

In many other places, I've reimplemented algorithms described in the
literature. These are too numerous to credit and thank here. The
original references are given in the comments of the code. However,
I've borrowed more than once from the following folks that I'd like to
be sure to thank: Steve Altschul, Pierre Baldi, Phillip Bucher, Warren
Gish, Steve and Jorja Henikoff, Anders Krogh, and Bill Pearson.

HMMER is primarily developed on GNU/Linux machines, but is tested on a
variety of hardware. Compaq, IBM, Intel, Sun Microsystems, Silicon
Graphics, Hewlett-Packard, and Paracel have provided the generous
hardware support that makes this possible. I owe a large debt to the
free software community for the development tools I use: an incomplete
list includes GNU gcc, gdb, emacs, and autoconf; Cygnus' and others'
egcs compiler; Conor Cahill's dbmalloc library; Bruce Perens'
ElectricFence; Armin Biere's ccmalloc; the cast of thousands that
develops CVS, the Concurrent Versioning System; Larry Wall's perl;
LaTeX and TeX from Leslie Lamport and Don Knuth; Nikos Drakos'
latex2html; Thomas Phelps' PolyglotMan; Linus Torvalds' Linux
operating system; and the folks at Red Hat Linux and Mandrake Linux.

Finally, I will cryptically thank Dave ``Mr. Frog'' Pare and Tom
``Chainsaw'' Ruschak for a totally unrelated free software product
that was historically instrumental in HMMER's development -- for
reasons that are best not discussed while sober.

\label{manualend}
\label{manualend}

% To create distributable/gitted 'distilled.bib' from lab's bibtex dbs:
%   # uncomment the {master,lab,books}
%   pdflatex main
%   bibdistill main.aux > distilled.bib
%   # restore the {distilled} 
% 
\nobibliography{distilled}
%\nobibliography{master,lab,books}

\end{document}



