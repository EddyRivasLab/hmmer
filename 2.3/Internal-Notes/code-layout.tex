\section{Notes on the layout of HMMER's source tree}

\subsection{History}

\begin{enumerate}
\item The code was first moved under CVS control in April 2000.
\end{enumerate}

\subsection{Module inclusion}

HMMER may rely on library code. An example is my SQUID library, which
provides sequence i/o and other functions. Future modules may be
provided by third parties; a possible example is the ``postprob''
package from Ian Holmes. Considerations that I took into account in
deciding how I include modules include:

\begin{description}
\item[Adhering to the GPL] Because modules may need to be removed from
certain non-GPL distributions because of license issues, any module's
code should be organized in an independent, easily removed
subdirectory. Makefiles should easily deal with the absence of
such encumbered modules.

\item[Ease of updating modules] A second reason in favor of
independent subdirectories: a module may be maintained by a third
party, rather than our CVS tree, and it may be preferable to update
such modules by rare complete replacements of their codebase, rather
than frequent CVS revs.

\item[Ease of configuration] A module must be compiled as part of a
single \texttt{./configure; make} command from the top level
directory.  Therefore each module has its own \texttt{Makefile.in}
(possibly created by me in addition to a third-party module).  This
\textt{Makefile.in} gets configuration variables in a standardized way
(since there's a single top-level configure script).

\item[No symbolic links during a build] Some systems allegedly don't
support symbolic linking. We therefore choose to put module object
files or .a files in a common directory -- for now, the \texttt{lib}
subdirectory. This is not a good strategy for supporting
multi-architecture 

\item[CVS tracking vs. proper attribution in version stamps] All
executables call \prog{Banner()} from \prog{squidcore.c} to print a
standardized header, including package name, version, release date,
copyright, and license info. Should a program from a module get
stamped with package-wide (e.g. HMMER) info, or with module-specific
(i.e. SQUID) information? Since I might release HMMER with
inter-release SQUID files, CVS tracking is my principal concern: my
modules get stamped as if they're all part of HMMER. With third-party
modules -- where presumably I'm using a stable release of that module
where someone else worries about rev control -- my principal concern
is proper attribution and retention of the owner's copyright and
license. In that (as yet hypothetical) case, module-specific programs
would retain module-specific licenses, copyright, etc.

\end{description}


\subsection{The configure script}


\subsection{The Makefile.in}

Targets expected in the module's \prog{Makefile.in}:

\begin{wideitem}
\item{\emprog{make all}}

Builds the object files and executables that the module provides.  If
this is my module, the file \prog{version.h} must first be created by
the top-level \prog{Makefile}, so that executables carry the
license/copyright banner of the master (HMMER) package. (If this is a
third party module, no versioning/copyright/license info is inherited
from the master package.)

\item{\emprog{make module}}
Builds a \prog{libxxx.a} module. Makes sure the .h files are 
ready, if any modifications need to be made. 

\item{\emprog{make install-module}}
Copies the \prog{.h} files to the build arena (currently 
\prog{\$top_srcdir/lib/}). Moves the \prog{libxxx.a} file
to the build arena.

\item{\emprog{make clean}}
Removes everything (e.g. \prog{.o} files) except configuration files
that were generated (e.g. the \prog{Makefile}).

\item{\emprog{make distclean}}
Removes everything that wasn't in the original source distribution.
\end{wideitem}